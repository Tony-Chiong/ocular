\section{Classification}\label{sec:randomforest}
By extracting the EEG feature vectors through the Filter Bank Common Spatial Patterns algorithm, we now train a classifier with the feature vectors as the training set.
We use the ensemble algorithm Random Forest as classification algorithm. We use the implementation available from the Scikit-Learn machine learning library for Python \cite{scikit-learn}.

The Random Forest technique works by splitting the training set of trials $T$ into $n$ subsets $\{t_1,…,t_n \ | \ t_i \subseteq S\}$ and trains $n$ decision trees for each subset $t_i$. The splitting is done randomly by drawing a bootstrap sample with replacement. The sets are constructed by uniformly choosing trials, with the possibility that one trial is drawn more than once. When classifying new trials, the Random Forest classifies on each of its decision trees and returns the mode result. Intuitively, the weak learners 'vote' on the result.

Each of the $n$ decision trees are constructed by randomly splitting the training subset $t_i$ on the features to obtain the training subset $t’_i \subset t_i$ containing the feature values of the randomly chosen features, in the given subset. The decision tree is then constructed according to a decision tree learning algorithm, typically constructing a node on the split with the highest information gain. No pruning takes place, as the randomness through voting, bootstrap sampling and random feature selection minimizes overfitting.

The performance of Random Forest classification models are mainly affected by the number of decision trees constructed and the number of features selected when splitting nodes. These parameters could be optimized through Bayesian Optimization. However the default parameters of the implementation in scikit-learn \citep{scikit-learn} as of version 0.17.1 are reasonable \citep{bernard2009influence}. Since we are only interested in the relative performance of the ocular artifact correction in \cref{sec:oacl}, we will not optimize over the Random Forest hyperparameters.