\begin{figure*}%[!hbtp]
	\centering
	\begin{adjustbox}{width=\textwidth}
		\newlength\figureheight
		\newlength\figurewidth
		\setlength\figureheight{6cm}
		\setlength\figurewidth{\textwidth}
		% This file was created by matplotlib2tikz v0.5.7.
% The lastest updates can be retrieved from
% 
% https://github.com/nschloe/matplotlib2tikz
% 
% where you can also submit bug reports and leavecomments.
\begin{tikzpicture}

\begin{groupplot}[group style={group size=1 by 2}]
\nextgroupplot[
xmin=0, xmax=750,
ymin=-35.009765625, ymax=30.908203125,
axis on top,
width=\figurewidth,
height=\figureheight,
legend entries={{\footnotesize raw signal: x(t)},{\footnotesize smoothed signal: s(t)},{\footnotesize artifacts: a(t)}},
legend cell align={left},
legend style={at={(0.97,0.03)}, anchor=south east}
]
\addplot [draw=red, mark=*, only marks] table {%
18 0
27 0
41 0
61 0
79 0
91 0
103 0
134 0
144 0
181 0
236 0
248 0
306 0
313 0
347 0
355 0
363 0
382 0
395 0
410 0
416 0
424 0
432 0
458 0
464 0
491 0
499 0
579 0
605 0
657 0
};
\addplot [blue, dashed]
table {%
0 -6.34765625
1 -3.369140625
2 -8.837890625
3 -9.912109375
4 -12.158203125
5 -17.822265625
6 -13.916015625
7 -15.673828125
8 -15.283203125
9 -14.501953125
10 -16.796875
11 -16.40625
12 -11.279296875
13 -10.83984375
14 -8.7890625
15 -11.328125
16 -11.181640625
17 -5.419921875
18 -5.46875
19 0.830078125
20 2.63671875
21 9.47265625
22 13.427734375
23 13.28125
24 11.03515625
25 6.884765625
26 -0.439453125
27 -5.224609375
28 -7.12890625
29 -6.73828125
30 -8.59375
31 -9.27734375
32 -14.990234375
33 -14.55078125
34 -11.572265625
35 -12.939453125
36 -10.302734375
37 -10.498046875
38 -8.203125
39 -12.451171875
40 -7.470703125
41 3.02734375
42 10.3515625
43 7.470703125
44 6.34765625
45 9.619140625
46 10.546875
47 6.15234375
48 8.30078125
49 4.00390625
50 2.099609375
51 -3.955078125
52 -2.197265625
53 6.34765625
54 4.98046875
55 7.470703125
56 6.640625
57 1.26953125
58 -0.5859375
59 -2.1484375
60 -0.537109375
61 1.46484375
62 3.22265625
63 -4.736328125
64 -3.22265625
65 -0.09765625
66 -1.513671875
67 -0.1953125
68 3.90625
69 3.3203125
70 -3.173828125
71 -3.955078125
72 -5.224609375
73 -1.708984375
74 -0.68359375
75 -4.150390625
76 -2.490234375
77 1.611328125
78 -4.150390625
79 -6.25
80 -4.541015625
81 1.513671875
82 -0.732421875
83 5.859375
84 10.25390625
85 9.033203125
86 3.80859375
87 4.4921875
88 9.814453125
89 8.30078125
90 5.615234375
91 -0.048828125
92 -9.5703125
93 -9.912109375
94 -11.474609375
95 -8.740234375
96 -3.076171875
97 0.09765625
98 -6.25
99 -10.791015625
100 -7.91015625
101 -5.126953125
102 -2.734375
103 1.3671875
104 2.9296875
105 3.7109375
106 2.880859375
107 4.736328125
108 13.0859375
109 9.375
110 10.986328125
111 13.818359375
112 5.2734375
113 3.41796875
114 10.791015625
115 14.697265625
116 9.375
117 12.451171875
118 5.517578125
119 0.634765625
120 1.708984375
121 1.806640625
122 6.884765625
123 10.888671875
124 5.712890625
125 6.0546875
126 3.564453125
127 -1.513671875
128 0.9765625
129 5.908203125
130 4.00390625
131 3.41796875
132 2.392578125
133 3.759765625
134 -0.68359375
135 -3.22265625
136 -7.763671875
137 -4.638671875
138 -3.076171875
139 -7.275390625
140 -4.78515625
141 -5.810546875
142 -6.494140625
143 -6.494140625
144 -3.369140625
145 -2.880859375
146 3.076171875
147 5.56640625
148 12.060546875
149 10.83984375
150 17.333984375
151 15.234375
152 6.494140625
153 7.958984375
154 8.0078125
155 6.15234375
156 8.88671875
157 4.98046875
158 5.615234375
159 3.80859375
160 0.9765625
161 7.12890625
162 10.107421875
163 15.087890625
164 15.234375
165 25.244140625
166 23.92578125
167 21.97265625
168 25.78125
169 24.560546875
170 30.76171875
171 30.908203125
172 16.50390625
173 22.16796875
174 22.216796875
175 15.673828125
176 13.818359375
177 16.259765625
178 12.255859375
179 1.513671875
180 -4.00390625
181 -8.740234375
182 -15.966796875
183 -9.66796875
184 -12.451171875
185 -5.95703125
186 -8.154296875
187 -10.15625
188 -6.982421875
189 -14.35546875
190 -10.15625
191 -12.060546875
192 -8.837890625
193 -3.02734375
194 -4.736328125
195 -5.712890625
196 -7.470703125
197 -5.029296875
198 -8.740234375
199 -8.7890625
200 -13.37890625
201 -15.8203125
202 -18.994140625
203 -20.654296875
204 -22.998046875
205 -23.291015625
206 -26.7578125
207 -35.009765625
208 -33.935546875
209 -28.3203125
210 -26.318359375
211 -24.51171875
212 -12.451171875
213 -8.740234375
214 -1.46484375
215 1.416015625
216 1.416015625
217 -0.9765625
218 -4.443359375
219 -0.341796875
220 0.048828125
221 3.515625
222 -1.26953125
223 0.48828125
224 -6.982421875
225 -8.056640625
226 -10.400390625
227 -11.62109375
228 -6.4453125
229 -7.080078125
230 -3.61328125
231 0.87890625
232 2.5390625
233 -0.732421875
234 1.904296875
235 -1.46484375
236 0.5859375
237 1.5625
238 -2.24609375
239 -3.369140625
240 -1.85546875
241 1.611328125
242 4.052734375
243 5.6640625
244 2.783203125
245 10.791015625
246 9.716796875
247 9.521484375
248 5.859375
249 5.908203125
250 -12.6953125
251 -12.890625
252 -13.76953125
253 -14.16015625
254 -3.3203125
255 -3.3203125
256 -1.171875
257 -4.4921875
258 -7.861328125
259 -6.15234375
260 -0.439453125
261 2.05078125
262 -3.90625
263 -2.5390625
264 -5.46875
265 -9.033203125
266 -13.0859375
267 -10.009765625
268 -12.353515625
269 -17.28515625
270 -14.892578125
271 -14.208984375
272 -12.59765625
273 -8.203125
274 -2.5390625
275 -2.880859375
276 -0.78125
277 -8.10546875
278 -9.765625
279 -13.916015625
280 -14.892578125
281 -15.283203125
282 -10.693359375
283 -16.064453125
284 -16.9921875
285 -16.9921875
286 -18.701171875
287 -14.84375
288 -10.9375
289 -3.61328125
290 -1.953125
291 1.85546875
292 0.09765625
293 -1.025390625
294 -3.857421875
295 -5.76171875
296 -13.18359375
297 -17.96875
298 -19.140625
299 -17.724609375
300 -18.1640625
301 -10.400390625
302 -2.978515625
303 -2.24609375
304 -0.732421875
305 -0.048828125
306 -4.443359375
307 -0.48828125
308 4.638671875
309 5.517578125
310 6.15234375
311 4.19921875
312 2.34375
313 -0.29296875
314 -4.6875
315 -0.927734375
316 -2.44140625
317 -6.591796875
318 -9.912109375
319 -9.66796875
320 -10.693359375
321 -12.5
322 -9.228515625
323 -7.2265625
324 -2.9296875
325 -3.857421875
326 0.341796875
327 2.294921875
328 -1.513671875
329 2.05078125
330 0.5859375
331 -0.634765625
332 -4.78515625
333 -3.076171875
334 -5.517578125
335 -4.248046875
336 -0.5859375
337 -3.3203125
338 3.076171875
339 0.9765625
340 -4.736328125
341 -5.908203125
342 -5.95703125
343 -2.685546875
344 -8.837890625
345 -8.10546875
346 -6.005859375
347 -5.859375
348 -2.24609375
349 4.248046875
350 8.0078125
351 10.400390625
352 11.474609375
353 7.32421875
354 0.146484375
355 -5.712890625
356 -6.494140625
357 -11.767578125
358 -9.130859375
359 -3.466796875
360 -3.02734375
361 0.830078125
362 3.90625
363 1.416015625
364 3.61328125
365 2.783203125
366 0.390625
367 -4.296875
368 6.982421875
369 1.513671875
370 4.39453125
371 3.564453125
372 0.68359375
373 0.732421875
374 2.978515625
375 2.099609375
376 2.880859375
377 1.611328125
378 11.572265625
379 3.125
380 4.58984375
381 3.125
382 -4.58984375
383 -2.1484375
384 -6.201171875
385 -6.298828125
386 -4.1015625
387 -6.15234375
388 -6.4453125
389 -2.44140625
390 -1.953125
391 0.87890625
392 -2.880859375
393 -2.978515625
394 -0.732421875
395 -1.3671875
396 -3.466796875
397 0.48828125
398 -4.6875
399 -0.146484375
400 5.95703125
401 12.109375
402 21.435546875
403 22.900390625
404 24.560546875
405 17.041015625
406 11.962890625
407 3.515625
408 1.025390625
409 -2.392578125
410 -5.078125
411 -6.005859375
412 -11.181640625
413 -14.35546875
414 -13.8671875
415 -10.986328125
416 -5.46875
417 4.4921875
418 14.697265625
419 13.28125
420 12.59765625
421 11.376953125
422 12.98828125
423 9.814453125
424 1.708984375
425 -3.90625
426 -9.521484375
427 -12.98828125
428 -21.923828125
429 -14.16015625
430 -11.865234375
431 -6.884765625
432 -1.806640625
433 -0.5859375
434 7.51953125
435 12.841796875
436 18.017578125
437 19.04296875
438 15.185546875
439 10.986328125
440 2.34375
441 -2.9296875
442 -5.95703125
443 -1.3671875
444 -0.87890625
445 2.44140625
446 3.7109375
447 4.931640625
448 8.0078125
449 10.888671875
450 12.841796875
451 16.2109375
452 10.986328125
453 18.017578125
454 10.9375
455 12.20703125
456 6.93359375
457 3.662109375
458 -5.126953125
459 -8.544921875
460 -8.935546875
461 -10.302734375
462 -13.18359375
463 -14.599609375
464 -8.0078125
465 -0.341796875
466 2.392578125
467 11.376953125
468 18.359375
469 18.115234375
470 12.255859375
471 15.966796875
472 8.935546875
473 6.689453125
474 5.56640625
475 2.83203125
476 4.296875
477 1.806640625
478 3.02734375
479 3.80859375
480 3.955078125
481 6.15234375
482 5.224609375
483 10.05859375
484 7.32421875
485 13.671875
486 13.818359375
487 12.40234375
488 8.203125
489 0.5859375
490 -2.44140625
491 -6.0546875
492 -8.837890625
493 -10.9375
494 -5.322265625
495 -7.666015625
496 -3.41796875
497 -0.390625
498 2.83203125
499 2.1484375
500 1.171875
501 3.41796875
502 -0.29296875
503 1.318359375
504 1.3671875
505 3.41796875
506 7.8125
507 5.46875
508 10.25390625
509 9.375
510 5.712890625
511 5.2734375
512 3.955078125
513 0.5859375
514 3.466796875
515 1.46484375
516 1.85546875
517 3.515625
518 1.806640625
519 2.197265625
520 5.712890625
521 9.08203125
522 12.841796875
523 17.28515625
524 10.546875
525 9.08203125
526 8.59375
527 5.2734375
528 6.0546875
529 11.5234375
530 16.9921875
531 27.685546875
532 19.482421875
533 18.5546875
534 24.70703125
535 19.921875
536 17.822265625
537 24.658203125
538 30.029296875
539 27.783203125
540 27.05078125
541 25.634765625
542 23.291015625
543 24.21875
544 27.392578125
545 22.75390625
546 25.78125
547 23.33984375
548 22.021484375
549 20.166015625
550 20.80078125
551 17.041015625
552 22.0703125
553 21.09375
554 23.14453125
555 23.291015625
556 23.095703125
557 20.3125
558 17.138671875
559 12.451171875
560 4.541015625
561 7.568359375
562 9.912109375
563 7.6171875
564 14.84375
565 17.724609375
566 21.484375
567 11.083984375
568 10.9375
569 13.232421875
570 8.69140625
571 9.814453125
572 14.2578125
573 12.744140625
574 12.3046875
575 9.912109375
576 9.27734375
577 3.173828125
578 3.3203125
579 -4.052734375
580 -8.69140625
581 -7.568359375
582 -14.2578125
583 -9.5703125
584 -4.78515625
585 -2.63671875
586 -3.955078125
587 -7.03125
588 -12.744140625
589 -14.501953125
590 -18.798828125
591 -13.0859375
592 -14.84375
593 -10.888671875
594 -8.349609375
595 -8.544921875
596 -8.3984375
597 -12.109375
598 -15.0390625
599 -14.111328125
600 -14.0625
601 -12.646484375
602 -12.40234375
603 -8.7890625
604 -0.439453125
605 -2.978515625
606 5.712890625
607 7.421875
608 8.3984375
609 11.62109375
610 9.47265625
611 12.646484375
612 8.7890625
613 12.01171875
614 10.205078125
615 14.35546875
616 12.3046875
617 15.234375
618 10.986328125
619 7.568359375
620 0.78125
621 -1.416015625
622 -0.5859375
623 0.48828125
624 1.26953125
625 -1.220703125
626 0.68359375
627 -0.634765625
628 10.7421875
629 6.494140625
630 11.03515625
631 9.130859375
632 2.685546875
633 -3.61328125
634 2.734375
635 8.0078125
636 4.4921875
637 7.568359375
638 4.19921875
639 4.19921875
640 -3.271484375
641 1.611328125
642 -0.146484375
643 -0.830078125
644 -0.634765625
645 -4.150390625
646 -3.61328125
647 3.759765625
648 1.611328125
649 13.525390625
650 13.623046875
651 14.111328125
652 8.935546875
653 7.958984375
654 0.341796875
655 1.46484375
656 3.515625
657 6.73828125
658 5.615234375
659 1.318359375
660 -8.154296875
661 -12.451171875
662 -17.431640625
663 -17.626953125
664 -9.9609375
665 -6.396484375
666 -11.9140625
667 -14.0625
668 -16.162109375
669 -17.333984375
670 -9.033203125
671 -8.88671875
672 -10.7421875
673 -9.130859375
674 -20.263671875
675 -14.2578125
676 -13.18359375
677 -5.810546875
678 -9.5703125
679 -8.544921875
680 -6.54296875
681 -6.15234375
682 -5.419921875
683 -0.048828125
684 2.783203125
685 -1.904296875
686 -5.029296875
687 -0.634765625
688 -2.978515625
689 -8.69140625
690 -6.4453125
691 -9.1796875
692 -8.447265625
693 -11.962890625
694 -12.6953125
695 -8.0078125
696 -6.884765625
697 -3.076171875
698 -3.7109375
699 -6.201171875
700 -10.888671875
701 -21.77734375
702 -23.046875
703 -23.095703125
704 -20.556640625
705 -20.458984375
706 -20.361328125
707 -23.095703125
708 -21.6796875
709 -20.60546875
710 -17.822265625
711 -11.328125
712 -8.154296875
713 -5.908203125
714 -6.591796875
715 -9.765625
716 1.806640625
717 3.61328125
718 0.244140625
719 0.244140625
720 -0.9765625
721 -2.783203125
722 -4.00390625
723 -9.765625
724 -8.10546875
725 -14.501953125
726 -19.43359375
727 -24.70703125
728 -19.091796875
729 -17.1875
730 -21.142578125
731 -19.23828125
732 -17.3828125
733 -19.384765625
734 -21.240234375
735 -19.287109375
736 -24.90234375
737 -33.0078125
738 -31.591796875
739 -31.25
740 -30.37109375
741 -26.66015625
742 -20.166015625
743 -18.75
744 -19.53125
745 -16.357421875
746 -14.697265625
747 -10.888671875
748 -15.869140625
749 -13.8671875
};
\addplot [green!50.0!black]
table {%
0 0
1 0
2 0
3 0
4 0
5 -12.2381036931818
6 -13.1525213068182
7 -13.8716264204545
8 -14.0536221590909
9 -13.9515269886364
10 -13.8760653409091
11 -13.2723721590909
12 -12.5
13 -11.572265625
14 -10.107421875
15 -8.54936079545454
16 -6.16122159090909
17 -3.44904119318182
18 -1.21626420454545
19 0.772372159090909
20 2.197265625
21 3.18714488636364
22 3.72869318181818
23 3.57333096590909
24 3.45791903409091
25 2.60120738636364
26 1.51811079545455
27 -0.705788352272727
28 -3.24928977272727
29 -5.50870028409091
30 -7.68821022727273
31 -9.25071022727273
32 -10.1651278409091
33 -10.4359019886364
34 -10.9197443181818
35 -10.986328125
36 -9.92986505681818
37 -8.14541903409091
38 -6.103515625
39 -4.20365767045455
40 -2.27716619318182
41 -0.142045454545455
42 1.35387073863636
43 3.06285511363636
44 4.17258522727273
45 5.49538352272727
46 5.81498579545455
47 5.34002130681818
48 4.97602982954546
49 4.74964488636364
50 4.85174005681818
51 4.58096590909091
52 3.73757102272727
53 3.125
54 2.17507102272727
55 1.76225142045455
56 1.70454545454545
57 2.35706676136364
58 2.12624289772727
59 1.25621448863636
60 0.794566761363636
61 -0.0221946022727273
62 -0.643643465909091
63 -0.403941761363636
64 -0.048828125
65 -0.142045454545455
66 -0.452769886363636
67 -1.06090198863636
68 -1.50923295454545
69 -1.14080255681818
70 -1.22514204545455
71 -1.44264914772727
72 -1.15855823863636
73 -1.51811079545455
74 -2.44140625
75 -3.15607244318182
76 -2.72993607954545
77 -2.43696732954545
78 -1.42933238636364
79 -0.341796875
80 0.541548295454545
81 1.26509232954545
82 1.89985795454545
83 2.64559659090909
84 3.77752130681818
85 4.85617897727273
86 5.26455965909091
87 4.25692471590909
88 3.42240767045455
89 1.84659090909091
90 0.119850852272727
91 -0.981001420454546
92 -1.318359375
93 -2.294921875
94 -4.16814630681818
95 -5.64186789772727
96 -6.61843039772727
97 -6.86257102272727
98 -5.86825284090909
99 -4.70081676136364
100 -3.3203125
101 -2.26384943181818
102 -1.55362215909091
103 -0.372869318181818
104 1.04758522727273
105 3.02734375
106 5.00266335227273
107 5.94815340909091
108 6.50745738636364
109 7.36416903409091
110 8.43394886363636
111 8.94886363636364
112 9.81889204545454
113 9.88991477272727
114 8.75799005681818
115 8.06107954545454
116 7.2265625
117 6.59623579545455
118 7.10671164772727
119 7.31534090909091
120 6.884765625
121 5.87269176136364
122 4.8828125
123 3.83966619318182
124 3.87517755681818
125 4.18146306818182
126 4.33682528409091
127 4.39009232954546
128 4.10600142045455
129 3.05397727272727
130 2.24165482954545
131 0.985440340909091
132 0.239701704545455
133 0.09765625
134 -0.652521306818182
135 -1.62464488636364
136 -2.51686789772727
137 -3.41796875
138 -4.22585227272727
139 -4.87393465909091
140 -5.07368607954546
141 -4.50106534090909
142 -3.28924005681818
143 -1.77112926136364
144 -0.506036931818182
145 1.73117897727273
146 3.55113636363636
147 4.66974431818182
148 5.98366477272727
149 7.30202414772727
150 8.16761363636364
151 9.23739346590909
152 9.41051136363636
153 9.41495028409091
154 8.66477272727273
155 7.76811079545455
156 6.84037642045455
157 6.37428977272727
158 7.15553977272727
159 7.81693892045455
160 9.38387784090909
161 10.9996448863636
162 12.1892755681818
163 14.0802556818182
164 15.8025568181818
165 18.2528409090909
166 20.9738991477273
167 21.826171875
168 22.9225852272727
169 23.5706676136364
170 23.6106178977273
171 22.5719105113636
172 21.875
173 20.9916548295455
174 18.7855113636364
175 16.1887428977273
176 12.59765625
177 8.33629261363636
178 5.95703125
179 2.80983664772727
180 0.248579545454545
181 -1.91761363636364
182 -4.09712357954546
183 -6.21004971590909
184 -8.62926136363636
185 -9.69016335227273
186 -10.4225852272727
187 -10.4314630681818
188 -9.25514914772727
189 -8.80681818181818
190 -8.19424715909091
191 -8.33185369318182
192 -8.04776278409091
193 -7.91903409090909
194 -8.08327414772727
195 -7.99449573863636
196 -8.50941051136364
197 -9.13973721590909
198 -10.2139559659091
199 -12.0294744318182
200 -13.7162642045455
201 -15.6294389204545
202 -18.1329900568182
203 -20.7608309659091
204 -22.5408380681818
205 -24.1344105113636
206 -25.146484375
207 -24.8401988636364
208 -23.9080255681818
209 -22.1635298295455
210 -19.9440696022727
211 -17.6979758522727
212 -15.3542258522727
213 -12.5754616477273
214 -9.521484375
215 -6.94247159090909
216 -4.23029119318182
217 -2.11736505681818
218 -0.941051136363636
219 -0.78125
220 -1.38050426136364
221 -2.45472301136364
222 -3.63991477272727
223 -4.13707386363636
224 -4.37677556818182
225 -4.67418323863636
226 -4.59872159090909
227 -4.6875
228 -4.638671875
229 -4.50994318181818
230 -4.00834517045455
231 -3.22265625
232 -2.13512073863636
233 -1.28284801136364
234 -1.00319602272727
235 -0.528231534090909
236 -0.0532670454545455
237 0.235262784090909
238 0.519353693181818
239 0.838955965909091
240 1.64683948863636
241 2.66335227272727
242 3.47567471590909
243 3.86629971590909
244 4.60759943181818
245 3.759765625
246 2.75656960227273
247 1.35830965909091
248 -0.297407670454545
249 -1.11416903409091
250 -1.66903409090909
251 -2.75656960227273
252 -4.04829545454546
253 -5.62855113636364
254 -6.72052556818182
255 -7.29758522727273
256 -5.95703125
257 -5.14026988636364
258 -4.11931818181818
259 -3.32919034090909
260 -3.84854403409091
261 -4.736328125
262 -5.53977272727273
263 -6.25443892045455
264 -7.11115056818182
265 -7.90571732954546
266 -9.15749289772727
267 -10.4891690340909
268 -10.8797940340909
269 -10.8797940340909
270 -10.64453125
271 -9.89435369318182
272 -9.44158380681818
273 -9.41938920454546
274 -9.56143465909091
275 -9.34392755681818
276 -9.37943892045454
277 -9.05983664772727
278 -9.375
279 -10.1740056818182
280 -11.4879261363636
281 -12.9261363636364
282 -14.2045454545455
283 -14.4620028409091
284 -13.9026988636364
285 -12.8151633522727
286 -11.2926136363636
287 -9.89435369318182
288 -9.01544744318182
289 -7.90571732954546
290 -6.884765625
291 -6.53852982954546
292 -6.47194602272727
293 -6.86257102272727
294 -7.47958096590909
295 -8.80237926136364
296 -9.5703125
297 -10.009765625
298 -10.2228338068182
299 -10.1962002840909
300 -9.84996448863636
301 -9.73011363636364
302 -8.57599431818182
303 -6.52077414772727
304 -4.27911931818182
305 -2.10848721590909
306 -0.0754616477272727
307 1.08309659090909
308 1.32723721590909
309 1.10529119318182
310 1.08753551136364
311 0.870028409090909
312 0.674715909090909
313 -0.181995738636364
314 -1.48259943181818
315 -2.95632102272727
316 -4.65198863636364
317 -5.87269176136364
318 -6.74272017045455
319 -6.982421875
320 -6.90696022727273
321 -6.79154829545455
322 -6.36097301136364
323 -5.89932528409091
324 -4.81178977272727
325 -3.87961647727273
326 -2.96519886363636
327 -2.26384943181818
328 -1.70454545454545
329 -1.54918323863636
330 -1.66903409090909
331 -1.37162642045455
332 -1.70454545454545
333 -1.63352272727273
334 -1.40713778409091
335 -2.02414772727273
336 -2.61452414772727
337 -3.09836647727273
338 -2.90749289772727
339 -3.43128551136364
340 -3.66654829545455
341 -3.82634943181818
342 -4.30575284090909
343 -4.20809659090909
344 -4.1015625
345 -3.46235795454545
346 -2.08629261363636
347 -0.506036931818182
348 0.701349431818182
349 0.958806818181818
350 1.24289772727273
351 1.38938210227273
352 0.865589488636364
353 0.568181818181818
354 0.457208806818182
355 -0.204190340909091
356 -0.856711647727273
357 -1.44708806818182
358 -2.36150568181818
359 -2.69886363636364
360 -2.45916193181818
361 -1.904296875
362 -1.70454545454545
363 0
364 0.967684659090909
365 1.68235085227273
366 2.28160511363636
367 2.26828835227273
368 1.97975852272727
369 2.12180397727273
370 1.98419744318182
371 1.99307528409091
372 2.10404829545455
373 3.54669744318182
374 3.19602272727273
375 3.47567471590909
376 3.36026278409091
377 2.61896306818182
378 2.36150568181818
379 1.73117897727273
380 0.887784090909091
381 0.324041193181818
382 -0.497159090909091
383 -1.22958096590909
384 -2.50355113636364
385 -2.96519886363636
386 -3.30255681818182
387 -3.84854403409091
388 -3.70205965909091
389 -3.57333096590909
390 -3.13387784090909
391 -2.87642045454545
392 -2.45916193181818
393 -2.32599431818182
394 -1.75337357954545
395 -0.989879261363636
396 0.288529829545455
397 2.15731534090909
398 4.50106534090909
399 7.00461647727273
400 8.62038352272727
401 9.83220880681818
402 10.4669744318182
403 10.5158025568182
404 10.7244318181818
405 10.2761008522727
406 9.18856534090909
407 7.07120028409091
408 3.81747159090909
409 0.474964488636364
410 -2.75656960227273
411 -4.80291193181818
412 -5.48206676136364
413 -4.46555397727273
414 -3.35138494318182
415 -1.98863636363636
416 -0.492720170454545
417 1.23401988636364
418 3.14275568181818
419 4.60316051136364
420 5.50870028409091
421 5.64186789772727
422 4.95827414772727
423 2.55681818181818
424 -0.0665838068181818
425 -2.35262784090909
426 -4.12375710227273
427 -5.322265625
428 -6.55628551136364
429 -6.76491477272727
430 -5.75284090909091
431 -3.759765625
432 -1.16299715909091
433 1.39825994318182
434 4.39009232954546
435 5.89044744318182
436 6.70276988636364
437 6.787109375
438 6.82705965909091
439 6.80042613636364
440 6.33877840909091
441 5.50870028409091
442 4.31906960227273
443 3.31587357954545
444 2.92524857954545
445 3.09392755681818
446 4.35458096590909
447 5.61967329545455
448 7.79918323863636
449 8.91779119318182
450 10.107421875
451 10.5158025568182
452 10.5113636363636
453 9.59694602272727
454 8.09215198863636
455 6.28995028409091
456 4.18590198863636
457 1.513671875
458 -0.812322443181818
459 -3.17826704545455
460 -4.20365767045455
461 -5.09588068181818
462 -4.69193892045455
463 -3.35582386363636
464 -1.24289772727273
465 0.648082386363636
466 2.91193181818182
467 4.66086647727273
468 6.46750710227273
469 8.30078125
470 9.28622159090909
471 9.70791903409091
472 9.65465198863636
473 8.89559659090909
474 7.57279829545455
475 6.28551136363636
476 5.73064630681818
477 4.75408380681818
478 4.85617897727273
479 4.91388494318182
480 5.65074573863636
481 6.64950284090909
482 7.38636363636364
483 7.96786221590909
484 7.74591619318182
485 7.177734375
486 6.26775568181818
487 4.90500710227273
488 3.43572443181818
489 2.03746448863636
490 0.674715909090909
491 -0.87890625
492 -2.17063210227273
493 -3.04066051136364
494 -3.59108664772727
495 -3.53781960227273
496 -3.00514914772727
497 -2.48135653409091
498 -1.55806107954545
499 -0.439453125
500 0.355113636363636
501 1.76225142045455
502 2.57013494318182
503 3.53781960227273
504 4.13263494318182
505 4.45667613636364
506 4.82954545454546
507 4.87837357954546
508 4.95827414772727
509 5.15358664772727
510 5.16246448863636
511 5.02041903409091
512 4.62979403409091
513 4.296875
514 3.564453125
515 3.23153409090909
516 3.53781960227273
517 4.22585227272727
518 5.43767755681818
519 6.34321732954546
520 6.85369318181818
521 7.50177556818182
522 7.8125
523 8.04332386363636
524 8.92666903409091
525 10.2716619318182
526 12.2691761363636
527 13.2146661931818
528 13.7340198863636
529 14.4087357954545
530 15.2610085227273
531 16.0555752840909
532 17.5159801136364
533 19.7665127840909
534 21.7418323863636
535 23.1534090909091
536 23.9390980113636
537 23.5395951704545
538 23.9701704545455
539 24.7736150568182
540 24.5960582386364
541 25.1287286931818
542 25.6303267045455
543 25.390625
544 24.4939630681818
545 23.8591974431818
546 22.94921875
547 22.6251775568182
548 22.4254261363636
549 22.3277698863636
550 21.9549005681818
551 21.9859730113636
552 21.4888139204545
553 20.9250710227273
554 20.0550426136364
555 18.6345880681818
556 17.431640625
557 16.7835582386364
558 15.4696377840909
559 14.9014559659091
560 14.4087357954545
561 14.2444957386364
562 13.1525213068182
563 12.3002485795455
564 11.9451349431818
565 11.6033380681818
566 12.0827414772727
567 12.6908735795455
568 12.9483309659091
569 13.3744673295455
570 12.9261363636364
571 12.158203125
572 10.4936079545455
573 9.78781960227273
574 8.42507102272727
575 6.43199573863636
576 4.95383522727273
577 2.76544744318182
578 0.599254261363636
579 -0.994318181818182
580 -2.35262784090909
581 -3.61328125
582 -5.09588068181818
583 -6.54296875
584 -8.16317471590909
585 -9.50372869318182
586 -9.90323153409091
587 -10.5646306818182
588 -10.2583451704545
589 -10.1473721590909
590 -10.4891690340909
591 -11.0129616477273
592 -11.7542613636364
593 -12.4822443181818
594 -12.6065340909091
595 -12.5665838068182
596 -12.0072798295455
597 -11.9451349431818
598 -11.3947088068182
599 -10.4447798295455
600 -9.95649857954546
601 -8.66033380681818
602 -7.22212357954546
603 -5.35777698863636
604 -2.93412642045455
605 -0.790127840909091
606 1.63796164772727
607 3.58664772727273
608 5.80610795454546
609 7.53284801136364
610 8.87784090909091
611 10.2672230113636
612 11.1328125
613 11.4568536931818
614 11.3813920454545
615 10.3959517045455
616 9.40607244318182
617 8.203125
618 7.44850852272727
619 6.47194602272727
620 5.43323863636364
621 4.19034090909091
622 3.01402698863636
623 2.60564630681818
624 2.197265625
625 2.51242897727273
626 3.271484375
627 3.64435369318182
628 3.369140625
629 3.57333096590909
630 4.18590198863636
631 4.70525568181818
632 5.33114346590909
633 5.77059659090909
634 5.17578125
635 4.28799715909091
636 3.43128551136364
637 2.587890625
638 2.26828835227273
639 2.5390625
640 1.91317471590909
641 0.856711647727273
642 0.790127840909091
643 0.248579545454545
644 1.09641335227273
645 1.953125
646 3.53338068181818
647 4.19921875
648 4.93607954545455
649 5.04261363636364
650 5.23348721590909
651 5.93039772727273
652 6.87144886363636
653 7.04012784090909
654 7.01349431818182
655 5.04261363636364
656 2.67223011363636
657 -0.1953125
658 -2.61008522727273
659 -4.23916903409091
660 -4.85174005681818
661 -6.06800426136364
662 -7.666015625
663 -9.74786931818182
664 -11.8341619318182
665 -12.7752130681818
666 -12.841796875
667 -12.6864346590909
668 -11.9318181818182
669 -12.1715198863636
670 -12.5621448863636
671 -13.1791548295455
672 -12.6242897727273
673 -12.2159090909091
674 -11.5234375
675 -10.5424360795455
676 -10.2805397727273
677 -9.96537642045454
678 -8.99325284090909
679 -7.91015625
680 -6.24112215909091
681 -5.40216619318182
682 -4.26136363636364
683 -4.00390625
684 -3.92400568181818
685 -3.73313210227273
686 -3.97283380681818
687 -4.18146306818182
688 -4.77627840909091
689 -5.92595880681818
690 -6.90696022727273
691 -7.35973011363636
692 -7.18217329545455
693 -7.46182528409091
694 -7.75479403409091
695 -7.95454545454546
696 -9.34836647727273
697 -10.6090198863636
698 -11.9406960227273
699 -12.7219460227273
700 -13.427734375
701 -14.55078125
702 -16.0245028409091
703 -17.7157315340909
704 -19.2515980113636
705 -20.3080610795455
706 -20.3480113636364
707 -19.1095525568182
708 -17.5514914772727
709 -16.0511363636364
710 -15.0701349431818
711 -13.0459872159091
712 -10.8664772727273
713 -8.74467329545454
714 -6.75159801136364
715 -4.96715198863636
716 -3.59996448863636
717 -2.93412642045455
718 -3.08061079545455
719 -3.28036221590909
720 -3.99946732954545
721 -4.87837357954546
722 -7.28870738636364
723 -9.35280539772727
724 -10.9375
725 -12.8817471590909
726 -14.5419034090909
727 -15.869140625
728 -17.2674005681818
729 -18.310546875
730 -19.3270596590909
731 -20.2725497159091
732 -21.5065696022727
733 -22.1324573863636
734 -23.2377485795455
735 -24.4362571022727
736 -24.9378551136364
737 -25.0221946022727
738 -25.146484375
739 -25.1598011363636
740 -24.7159090909091
741 -24.2986505681818
742 -23.0246803977273
743 -21.4666193181818
744 -19.8552911931818
745 0
746 0
747 0
748 0
749 0
};
\addplot [red]
table {%
0 0
1 0
2 0
3 0
4 0
5 0
6 0
7 0
8 0
9 0
10 0
11 0
12 0
13 0
14 0
15 0
16 0
17 0
18 0
19 0
20 0
21 0
22 0
23 0
24 0
25 0
26 0
27 0
28 0
29 0
30 0
31 0
32 0
33 0
34 0
35 0
36 0
37 0
38 0
39 0
40 0
41 0
42 0
43 0
44 0
45 0
46 0
47 0
48 0
49 0
50 0
51 0
52 0
53 0
54 0
55 0
56 0
57 0
58 0
59 0
60 0
61 0
62 0
63 0
64 0
65 0
66 0
67 0
68 0
69 0
70 0
71 0
72 0
73 0
74 0
75 0
76 0
77 0
78 0
79 0
80 0
81 0
82 0
83 0
84 0
85 0
86 0
87 0
88 0
89 0
90 0
91 0
92 0
93 0
94 0
95 0
96 0
97 0
98 0
99 0
100 0
101 0
102 0
103 0
104 0
105 0
106 0
107 0
108 0
109 0
110 0
111 0
112 0
113 0
114 0
115 0
116 0
117 0
118 0
119 0
120 0
121 0
122 0
123 0
124 0
125 0
126 0
127 0
128 0
129 0
130 0
131 0
132 0
133 0
134 0
135 0
136 0
137 0
138 0
139 0
140 0
141 0
142 0
143 0
144 0
145 1.73117897727273
146 3.55113636363636
147 4.66974431818182
148 5.98366477272727
149 7.30202414772727
150 8.16761363636364
151 9.23739346590909
152 9.41051136363636
153 9.41495028409091
154 8.66477272727273
155 7.76811079545455
156 6.84037642045455
157 6.37428977272727
158 7.15553977272727
159 7.81693892045455
160 9.38387784090909
161 10.9996448863636
162 12.1892755681818
163 14.0802556818182
164 15.8025568181818
165 18.2528409090909
166 20.9738991477273
167 21.826171875
168 22.9225852272727
169 23.5706676136364
170 23.6106178977273
171 22.5719105113636
172 21.875
173 20.9916548295455
174 18.7855113636364
175 16.1887428977273
176 12.59765625
177 8.33629261363636
178 5.95703125
179 2.80983664772727
180 0.248579545454545
181 0
182 0
183 0
184 0
185 0
186 0
187 0
188 0
189 0
190 0
191 0
192 0
193 0
194 0
195 0
196 0
197 0
198 0
199 0
200 0
201 0
202 0
203 0
204 0
205 0
206 0
207 0
208 0
209 0
210 0
211 0
212 0
213 0
214 0
215 0
216 0
217 0
218 0
219 0
220 0
221 0
222 0
223 0
224 0
225 0
226 0
227 0
228 0
229 0
230 0
231 0
232 0
233 0
234 0
235 0
236 0
237 0
238 0
239 0
240 0
241 0
242 0
243 0
244 0
245 0
246 0
247 0
248 0
249 0
250 0
251 0
252 0
253 0
254 0
255 0
256 0
257 0
258 0
259 0
260 0
261 0
262 0
263 0
264 0
265 0
266 0
267 0
268 0
269 0
270 0
271 0
272 0
273 0
274 0
275 0
276 0
277 0
278 0
279 0
280 0
281 0
282 0
283 0
284 0
285 0
286 0
287 0
288 0
289 0
290 0
291 0
292 0
293 0
294 0
295 0
296 0
297 0
298 0
299 0
300 0
301 0
302 0
303 0
304 0
305 0
306 0
307 0
308 0
309 0
310 0
311 0
312 0
313 0
314 0
315 0
316 0
317 0
318 0
319 0
320 0
321 0
322 0
323 0
324 0
325 0
326 0
327 0
328 0
329 0
330 0
331 0
332 0
333 0
334 0
335 0
336 0
337 0
338 0
339 0
340 0
341 0
342 0
343 0
344 0
345 0
346 0
347 0
348 0
349 0
350 0
351 0
352 0
353 0
354 0
355 0
356 0
357 0
358 0
359 0
360 0
361 0
362 0
363 0
364 0
365 0
366 0
367 0
368 0
369 0
370 0
371 0
372 0
373 0
374 0
375 0
376 0
377 0
378 0
379 0
380 0
381 0
382 0
383 0
384 0
385 0
386 0
387 0
388 0
389 0
390 0
391 0
392 0
393 0
394 0
395 0
396 0
397 0
398 0
399 0
400 0
401 0
402 0
403 0
404 0
405 0
406 0
407 0
408 0
409 0
410 0
411 0
412 0
413 0
414 0
415 0
416 0
417 0
418 0
419 0
420 0
421 0
422 0
423 0
424 0
425 0
426 0
427 0
428 0
429 0
430 0
431 0
432 0
433 0
434 0
435 0
436 0
437 0
438 0
439 0
440 0
441 0
442 0
443 0
444 0
445 0
446 0
447 0
448 0
449 0
450 0
451 0
452 0
453 0
454 0
455 0
456 0
457 0
458 0
459 0
460 0
461 0
462 0
463 0
464 0
465 0
466 0
467 0
468 0
469 0
470 0
471 0
472 0
473 0
474 0
475 0
476 0
477 0
478 0
479 0
480 0
481 0
482 0
483 0
484 0
485 0
486 0
487 0
488 0
489 0
490 0
491 0
492 0
493 0
494 0
495 0
496 0
497 0
498 0
499 0
500 0
501 0
502 0
503 0
504 0
505 0
506 0
507 0
508 0
509 0
510 0
511 0
512 0
513 0
514 0
515 0
516 0
517 0
518 0
519 0
520 0
521 0
522 0
523 0
524 0
525 0
526 0
527 0
528 0
529 0
530 0
531 0
532 0
533 0
534 0
535 0
536 0
537 0
538 0
539 0
540 0
541 0
542 0
543 0
544 0
545 0
546 0
547 0
548 0
549 0
550 0
551 0
552 0
553 0
554 0
555 0
556 0
557 0
558 0
559 0
560 0
561 0
562 0
563 0
564 0
565 0
566 0
567 0
568 0
569 0
570 0
571 0
572 0
573 0
574 0
575 0
576 0
577 0
578 0
579 0
580 0
581 0
582 0
583 0
584 0
585 0
586 0
587 0
588 0
589 0
590 0
591 0
592 0
593 0
594 0
595 0
596 0
597 0
598 0
599 0
600 0
601 0
602 0
603 0
604 0
605 0
606 0
607 0
608 0
609 0
610 0
611 0
612 0
613 0
614 0
615 0
616 0
617 0
618 0
619 0
620 0
621 0
622 0
623 0
624 0
625 0
626 0
627 0
628 0
629 0
630 0
631 0
632 0
633 0
634 0
635 0
636 0
637 0
638 0
639 0
640 0
641 0
642 0
643 0
644 0
645 0
646 0
647 0
648 0
649 0
650 0
651 0
652 0
653 0
654 0
655 0
656 0
657 0
658 0
659 0
660 0
661 0
662 0
663 0
664 0
665 0
666 0
667 0
668 0
669 0
670 0
671 0
672 0
673 0
674 0
675 0
676 0
677 0
678 0
679 0
680 0
681 0
682 0
683 0
684 0
685 0
686 0
687 0
688 0
689 0
690 0
691 0
692 0
693 0
694 0
695 0
696 0
697 0
698 0
699 0
700 0
701 0
702 0
703 0
704 0
705 0
706 0
707 0
708 0
709 0
710 0
711 0
712 0
713 0
714 0
715 0
716 0
717 0
718 0
719 0
720 0
721 0
722 0
723 0
724 0
725 0
726 0
727 0
728 0
729 0
730 0
731 0
732 0
733 0
734 0
735 0
736 0
737 0
738 0
739 0
740 0
741 0
742 0
743 0
744 0
};
\nextgroupplot[
xlabel={time (t)},
ylabel={amplitude},
xmin=0, xmax=750,
ymin=-38.57421875, ymax=117.67578125,
axis on top,
width=\figurewidth,
height=\figureheight,
legend entries={{raw signal: x(t)},{smoothed signal: s(t)},{artifacts: a(t)}},
legend entries={{EOG 1},{EOG 2},{EOG 3}},
legend cell align={left},
legend style={at={(0.97,0.03)}, anchor=south east}
]
\addplot [red]
table {%
0 6.34765625
1 6.8359375
2 3.41796875
3 -1.46484375
4 0.48828125
5 -1.46484375
6 2.9296875
7 0.9765625
8 2.44140625
9 -0.48828125
10 -1.953125
11 -4.8828125
12 -1.46484375
13 3.90625
14 1.46484375
15 2.9296875
16 7.8125
17 8.30078125
18 8.30078125
19 5.859375
20 9.27734375
21 3.90625
22 5.859375
23 7.32421875
24 7.32421875
25 9.27734375
26 8.30078125
27 6.8359375
28 8.7890625
29 8.30078125
30 -0.48828125
31 5.859375
32 -3.41796875
33 -0.9765625
34 -0.9765625
35 -0.48828125
36 -3.90625
37 -4.39453125
38 -2.9296875
39 -2.44140625
40 2.9296875
41 9.765625
42 12.6953125
43 8.7890625
44 14.16015625
45 11.23046875
46 11.23046875
47 9.765625
48 17.578125
49 15.625
50 10.7421875
51 10.25390625
52 11.71875
53 15.13671875
54 15.13671875
55 15.625
56 19.04296875
57 11.71875
58 7.8125
59 7.8125
60 0.48828125
61 7.32421875
62 7.32421875
63 6.34765625
64 16.11328125
65 6.8359375
66 9.765625
67 8.7890625
68 7.32421875
69 7.8125
70 6.8359375
71 5.37109375
72 4.39453125
73 7.8125
74 5.859375
75 1.46484375
76 3.90625
77 5.859375
78 1.953125
79 1.46484375
80 0.9765625
81 -1.46484375
82 4.8828125
83 3.90625
84 5.37109375
85 2.44140625
86 5.859375
87 0.48828125
88 10.7421875
89 5.859375
90 7.8125
91 7.8125
92 0.48828125
93 1.46484375
94 2.9296875
95 2.44140625
96 1.953125
97 5.859375
98 6.34765625
99 1.953125
100 5.37109375
101 3.90625
102 -1.953125
103 3.90625
104 4.8828125
105 9.27734375
106 7.32421875
107 1.953125
108 4.8828125
109 7.32421875
110 7.8125
111 4.39453125
112 -1.953125
113 -2.44140625
114 1.46484375
115 -9.27734375
116 -7.32421875
117 3.41796875
118 0.48828125
119 -0.9765625
120 -1.46484375
121 -2.44140625
122 -7.32421875
123 -7.32421875
124 -11.71875
125 -3.41796875
126 -1.953125
127 -4.8828125
128 -1.46484375
129 -2.9296875
130 -3.90625
131 -4.8828125
132 -2.9296875
133 0.9765625
134 -5.37109375
135 -7.8125
136 -3.90625
137 -6.8359375
138 -4.39453125
139 -0.9765625
140 -1.953125
141 -4.39453125
142 -5.37109375
143 -6.34765625
144 0.9765625
145 -2.9296875
146 2.44140625
147 -1.953125
148 0
149 -4.8828125
150 0.9765625
151 1.46484375
152 -2.44140625
153 4.39453125
154 4.39453125
155 -0.48828125
156 1.46484375
157 3.90625
158 0
159 -1.46484375
160 -1.46484375
161 4.39453125
162 3.90625
163 2.44140625
164 -3.41796875
165 0.48828125
166 -1.46484375
167 -3.90625
168 1.953125
169 -2.9296875
170 4.39453125
171 1.953125
172 -5.859375
173 5.859375
174 8.30078125
175 5.859375
176 5.859375
177 4.8828125
178 9.765625
179 1.953125
180 0
181 5.859375
182 2.44140625
183 -1.953125
184 5.859375
185 3.41796875
186 0.48828125
187 5.859375
188 12.6953125
189 -0.48828125
190 5.859375
191 6.8359375
192 -0.48828125
193 4.8828125
194 6.8359375
195 0.48828125
196 2.44140625
197 4.39453125
198 5.37109375
199 1.953125
200 1.46484375
201 6.8359375
202 0
203 6.8359375
204 1.46484375
205 -2.44140625
206 -0.9765625
207 -4.39453125
208 -6.8359375
209 -2.44140625
210 -4.39453125
211 -0.9765625
212 8.30078125
213 1.46484375
214 10.25390625
215 11.23046875
216 8.7890625
217 7.32421875
218 7.8125
219 14.16015625
220 11.23046875
221 9.765625
222 8.7890625
223 11.71875
224 2.9296875
225 8.30078125
226 2.44140625
227 7.8125
228 2.44140625
229 12.6953125
230 6.8359375
231 15.625
232 12.6953125
233 6.8359375
234 9.765625
235 5.37109375
236 5.37109375
237 8.30078125
238 3.41796875
239 -3.41796875
240 3.90625
241 2.9296875
242 7.32421875
243 5.859375
244 8.30078125
245 7.8125
246 5.859375
247 7.8125
248 2.44140625
249 5.859375
250 -6.8359375
251 0
252 -5.37109375
253 -0.48828125
254 5.859375
255 5.37109375
256 11.71875
257 4.39453125
258 1.953125
259 -0.48828125
260 1.953125
261 1.46484375
262 1.953125
263 6.34765625
264 1.953125
265 2.9296875
266 10.7421875
267 9.27734375
268 9.765625
269 4.39453125
270 3.90625
271 5.37109375
272 6.34765625
273 6.34765625
274 16.6015625
275 17.08984375
276 10.7421875
277 13.18359375
278 4.8828125
279 3.90625
280 2.44140625
281 7.32421875
282 7.8125
283 5.859375
284 7.32421875
285 2.9296875
286 9.27734375
287 4.8828125
288 3.90625
289 1.953125
290 2.9296875
291 6.34765625
292 2.44140625
293 5.859375
294 4.39453125
295 4.39453125
296 4.39453125
297 0.48828125
298 0.48828125
299 0.48828125
300 2.44140625
301 9.765625
302 0
303 -1.46484375
304 -6.34765625
305 3.41796875
306 -0.48828125
307 6.34765625
308 10.7421875
309 13.18359375
310 9.27734375
311 4.39453125
312 11.23046875
313 10.25390625
314 6.8359375
315 9.765625
316 7.8125
317 5.37109375
318 -0.48828125
319 12.6953125
320 -0.48828125
321 2.9296875
322 4.39453125
323 3.90625
324 3.41796875
325 -2.44140625
326 4.8828125
327 10.25390625
328 5.859375
329 14.6484375
330 12.6953125
331 10.7421875
332 4.39453125
333 3.90625
334 3.90625
335 -2.44140625
336 8.7890625
337 -2.9296875
338 4.39453125
339 0.9765625
340 5.37109375
341 4.8828125
342 6.34765625
343 -0.48828125
344 2.44140625
345 1.46484375
346 4.39453125
347 3.41796875
348 1.46484375
349 8.7890625
350 8.30078125
351 10.7421875
352 13.18359375
353 8.7890625
354 9.765625
355 0.9765625
356 2.44140625
357 1.46484375
358 1.953125
359 5.37109375
360 1.953125
361 5.37109375
362 8.7890625
363 9.765625
364 9.27734375
365 2.44140625
366 5.859375
367 -0.48828125
368 8.30078125
369 8.30078125
370 11.23046875
371 6.8359375
372 8.7890625
373 1.953125
374 5.37109375
375 14.16015625
376 11.71875
377 9.27734375
378 16.11328125
379 10.25390625
380 4.39453125
381 5.859375
382 4.39453125
383 13.671875
384 3.41796875
385 10.7421875
386 9.27734375
387 1.953125
388 -3.90625
389 2.44140625
390 3.90625
391 0.48828125
392 -1.46484375
393 0.48828125
394 0.9765625
395 5.37109375
396 2.44140625
397 3.90625
398 -1.46484375
399 1.46484375
400 -1.46484375
401 9.765625
402 4.39453125
403 12.6953125
404 16.6015625
405 7.32421875
406 9.765625
407 0
408 0.9765625
409 -1.46484375
410 -1.46484375
411 -1.953125
412 -1.46484375
413 0.9765625
414 -1.953125
415 1.46484375
416 3.41796875
417 5.859375
418 6.8359375
419 1.46484375
420 6.34765625
421 5.859375
422 3.41796875
423 11.71875
424 0.9765625
425 1.953125
426 -2.44140625
427 -4.8828125
428 -7.32421875
429 3.90625
430 0
431 4.8828125
432 7.8125
433 2.44140625
434 3.41796875
435 4.39453125
436 5.37109375
437 7.32421875
438 5.37109375
439 5.859375
440 4.39453125
441 11.71875
442 4.8828125
443 8.7890625
444 7.8125
445 10.25390625
446 8.30078125
447 6.8359375
448 9.765625
449 9.765625
450 6.8359375
451 7.8125
452 -2.9296875
453 1.46484375
454 1.46484375
455 0.48828125
456 -2.44140625
457 4.39453125
458 3.90625
459 2.9296875
460 2.9296875
461 2.44140625
462 2.9296875
463 0
464 -3.41796875
465 0.48828125
466 -4.39453125
467 5.37109375
468 7.32421875
469 7.32421875
470 4.39453125
471 4.8828125
472 -5.37109375
473 -2.44140625
474 4.39453125
475 2.44140625
476 9.765625
477 13.671875
478 12.20703125
479 2.9296875
480 1.953125
481 0.9765625
482 2.44140625
483 1.953125
484 -2.44140625
485 3.90625
486 5.859375
487 1.46484375
488 1.953125
489 5.859375
490 8.30078125
491 2.9296875
492 0
493 -5.37109375
494 2.9296875
495 -3.90625
496 1.953125
497 1.953125
498 5.37109375
499 4.8828125
500 2.44140625
501 8.30078125
502 6.8359375
503 5.859375
504 7.8125
505 3.41796875
506 8.30078125
507 3.90625
508 4.39453125
509 7.8125
510 2.9296875
511 9.27734375
512 -0.9765625
513 1.46484375
514 7.8125
515 3.90625
516 4.8828125
517 1.953125
518 -1.953125
519 -2.9296875
520 0.9765625
521 9.765625
522 3.90625
523 -2.9296875
524 -6.8359375
525 1.46484375
526 6.8359375
527 5.859375
528 13.18359375
529 10.7421875
530 -1.46484375
531 2.44140625
532 -2.44140625
533 -1.46484375
534 3.41796875
535 3.41796875
536 -1.953125
537 0.9765625
538 2.9296875
539 5.37109375
540 3.41796875
541 7.32421875
542 4.39453125
543 4.8828125
544 12.20703125
545 6.34765625
546 11.23046875
547 5.859375
548 7.32421875
549 10.7421875
550 8.7890625
551 0.48828125
552 5.37109375
553 9.27734375
554 8.7890625
555 13.671875
556 10.25390625
557 5.859375
558 11.23046875
559 0.48828125
560 -3.90625
561 3.90625
562 -4.39453125
563 -3.41796875
564 6.34765625
565 -1.46484375
566 12.20703125
567 5.37109375
568 4.8828125
569 6.34765625
570 0.9765625
571 6.8359375
572 4.39453125
573 7.32421875
574 8.30078125
575 12.20703125
576 9.27734375
577 3.41796875
578 12.20703125
579 -3.41796875
580 -10.25390625
581 0
582 5.37109375
583 6.34765625
584 11.23046875
585 12.20703125
586 9.27734375
587 -1.46484375
588 1.46484375
589 3.90625
590 2.44140625
591 2.9296875
592 1.953125
593 4.39453125
594 5.37109375
595 3.41796875
596 10.25390625
597 0.9765625
598 0.9765625
599 -0.9765625
600 -0.48828125
601 2.9296875
602 2.44140625
603 0.9765625
604 7.32421875
605 4.39453125
606 6.34765625
607 4.39453125
608 5.859375
609 5.859375
610 1.953125
611 10.25390625
612 1.953125
613 5.37109375
614 4.39453125
615 3.90625
616 -0.9765625
617 -0.9765625
618 -5.37109375
619 1.46484375
620 -0.48828125
621 -1.953125
622 2.9296875
623 -0.9765625
624 1.46484375
625 -1.953125
626 2.9296875
627 -2.9296875
628 5.859375
629 6.8359375
630 4.39453125
631 9.27734375
632 13.18359375
633 0
634 -0.48828125
635 8.30078125
636 3.41796875
637 3.90625
638 5.859375
639 1.953125
640 -1.953125
641 3.90625
642 2.9296875
643 -0.48828125
644 2.44140625
645 2.44140625
646 -1.46484375
647 4.8828125
648 -1.46484375
649 4.39453125
650 6.8359375
651 1.953125
652 -1.953125
653 1.46484375
654 8.7890625
655 0
656 4.8828125
657 6.8359375
658 6.34765625
659 11.23046875
660 5.859375
661 4.8828125
662 0
663 0.9765625
664 0.9765625
665 4.39453125
666 8.7890625
667 3.90625
668 1.46484375
669 -0.48828125
670 6.34765625
671 6.34765625
672 -0.9765625
673 9.27734375
674 -0.9765625
675 4.39453125
676 -1.46484375
677 6.34765625
678 -1.953125
679 3.41796875
680 -0.9765625
681 10.25390625
682 12.20703125
683 12.6953125
684 9.765625
685 4.8828125
686 -1.953125
687 4.8828125
688 4.8828125
689 -1.46484375
690 6.34765625
691 0
692 -4.39453125
693 5.859375
694 7.32421875
695 6.8359375
696 4.8828125
697 8.7890625
698 6.34765625
699 4.8828125
700 4.8828125
701 0.48828125
702 -6.34765625
703 0.48828125
704 -0.48828125
705 -4.8828125
706 2.44140625
707 -2.44140625
708 -5.859375
709 -4.39453125
710 -2.9296875
711 -0.9765625
712 -2.9296875
713 1.46484375
714 2.44140625
715 -0.48828125
716 5.37109375
717 7.32421875
718 -3.41796875
719 4.39453125
720 -1.953125
721 3.90625
722 4.8828125
723 -4.39453125
724 3.41796875
725 -6.8359375
726 -0.48828125
727 -5.37109375
728 4.39453125
729 7.8125
730 3.41796875
731 6.8359375
732 7.8125
733 0
734 2.44140625
735 2.44140625
736 -1.953125
737 -3.41796875
738 -4.39453125
739 -6.34765625
740 -7.32421875
741 0
742 -0.48828125
743 -0.9765625
744 2.9296875
745 0.9765625
746 0
747 -0.9765625
748 3.90625
749 -2.9296875
};
\addplot [blue]
table {%
0 5.859375
1 3.41796875
2 2.44140625
3 0.9765625
4 -2.9296875
5 -9.765625
6 2.9296875
7 1.953125
8 3.41796875
9 0.48828125
10 -7.32421875
11 -10.25390625
12 -9.27734375
13 -0.9765625
14 2.44140625
15 3.90625
16 6.34765625
17 7.32421875
18 -2.44140625
19 4.39453125
20 7.32421875
21 5.37109375
22 8.7890625
23 15.625
24 6.8359375
25 3.41796875
26 7.32421875
27 4.8828125
28 7.8125
29 10.25390625
30 -0.9765625
31 -0.9765625
32 -6.34765625
33 -7.8125
34 -5.859375
35 0.48828125
36 -2.9296875
37 -9.27734375
38 -6.8359375
39 -0.48828125
40 6.8359375
41 13.18359375
42 22.4609375
43 16.11328125
44 10.25390625
45 19.53125
46 21.484375
47 17.578125
48 30.2734375
49 20.01953125
50 18.5546875
51 13.671875
52 11.23046875
53 25.390625
54 25.390625
55 18.06640625
56 20.01953125
57 17.08984375
58 9.765625
59 10.7421875
60 7.8125
61 14.6484375
62 9.765625
63 2.44140625
64 9.27734375
65 3.41796875
66 3.41796875
67 11.71875
68 5.37109375
69 9.27734375
70 8.7890625
71 5.37109375
72 0.9765625
73 11.23046875
74 10.7421875
75 4.39453125
76 7.8125
77 12.6953125
78 11.71875
79 3.41796875
80 6.8359375
81 4.8828125
82 17.578125
83 24.4140625
84 26.3671875
85 19.53125
86 17.08984375
87 6.8359375
88 8.30078125
89 13.671875
90 13.18359375
91 14.6484375
92 3.90625
93 4.8828125
94 5.859375
95 -0.9765625
96 0
97 6.34765625
98 0.9765625
99 -0.9765625
100 5.859375
101 5.859375
102 4.39453125
103 0.48828125
104 5.859375
105 3.41796875
106 8.7890625
107 3.90625
108 9.27734375
109 12.20703125
110 8.7890625
111 12.20703125
112 -0.48828125
113 -8.30078125
114 9.27734375
115 11.71875
116 10.25390625
117 19.04296875
118 20.01953125
119 15.13671875
120 4.39453125
121 5.37109375
122 10.7421875
123 14.16015625
124 8.30078125
125 15.625
126 10.7421875
127 8.7890625
128 9.765625
129 10.7421875
130 11.23046875
131 2.44140625
132 6.8359375
133 5.859375
134 1.46484375
135 -5.37109375
136 -4.8828125
137 -7.32421875
138 -3.90625
139 -3.90625
140 -8.7890625
141 -10.7421875
142 -5.37109375
143 -9.27734375
144 -4.39453125
145 -2.44140625
146 3.90625
147 0.9765625
148 15.625
149 9.765625
150 14.16015625
151 17.578125
152 9.27734375
153 21.97265625
154 27.34375
155 28.3203125
156 38.0859375
157 44.921875
158 46.38671875
159 41.015625
160 50.78125
161 69.82421875
162 83.0078125
163 85.44921875
164 96.19140625
165 109.375
166 106.4453125
167 107.421875
168 117.67578125
169 107.421875
170 115.72265625
171 113.76953125
172 93.26171875
173 96.6796875
174 88.8671875
175 79.1015625
176 68.84765625
177 63.96484375
178 60.546875
179 40.52734375
180 29.296875
181 22.4609375
182 15.625
183 13.671875
184 5.37109375
185 7.32421875
186 1.953125
187 -3.90625
188 -3.41796875
189 -8.7890625
190 -12.20703125
191 -13.671875
192 -17.578125
193 -11.71875
194 -15.625
195 -17.578125
196 -21.97265625
197 -24.90234375
198 -23.4375
199 -31.25
200 -25.87890625
201 -23.92578125
202 -26.85546875
203 -27.34375
204 -33.203125
205 -38.57421875
206 -38.0859375
207 -38.57421875
208 -36.62109375
209 -30.76171875
210 -35.64453125
211 -34.1796875
212 -28.3203125
213 -25.87890625
214 -15.625
215 -17.08984375
216 -14.6484375
217 -19.04296875
218 -19.04296875
219 -16.6015625
220 -14.6484375
221 -16.11328125
222 -17.08984375
223 -10.25390625
224 -19.04296875
225 -19.04296875
226 -17.08984375
227 -18.5546875
228 -21.484375
229 -16.11328125
230 -16.6015625
231 -5.37109375
232 -5.37109375
233 -8.7890625
234 -11.23046875
235 -7.32421875
236 -4.8828125
237 -6.34765625
238 -13.18359375
239 -14.16015625
240 -15.625
241 -14.16015625
242 -13.671875
243 -11.23046875
244 -18.5546875
245 -7.8125
246 -17.578125
247 -11.23046875
248 -15.13671875
249 -11.23046875
250 -31.25
251 -25.87890625
252 -26.85546875
253 -27.83203125
254 -17.578125
255 -21.484375
256 -16.6015625
257 -19.53125
258 -19.04296875
259 -18.5546875
260 -13.671875
261 -17.08984375
262 -24.4140625
263 -19.53125
264 -18.5546875
265 -18.5546875
266 -19.04296875
267 -16.11328125
268 -16.11328125
269 -27.34375
270 -20.5078125
271 -19.04296875
272 -21.484375
273 -19.04296875
274 -1.46484375
275 -5.859375
276 -4.39453125
277 -12.6953125
278 -16.11328125
279 -21.484375
280 -20.5078125
281 -16.6015625
282 -12.6953125
283 -17.08984375
284 -11.71875
285 -20.5078125
286 -22.94921875
287 -20.5078125
288 -20.99609375
289 -17.578125
290 -17.08984375
291 -14.16015625
292 -0.9765625
293 -14.16015625
294 -9.765625
295 -7.8125
296 -14.16015625
297 -18.06640625
298 -19.53125
299 -16.6015625
300 -18.06640625
301 -12.20703125
302 -13.671875
303 -13.671875
304 -16.6015625
305 -7.32421875
306 -15.625
307 -6.34765625
308 -1.46484375
309 -0.48828125
310 1.46484375
311 -7.32421875
312 0.48828125
313 -2.44140625
314 -9.765625
315 -4.8828125
316 -1.953125
317 -4.8828125
318 -9.765625
319 -4.39453125
320 -15.13671875
321 -13.671875
322 -11.23046875
323 -10.7421875
324 -3.90625
325 -6.34765625
326 2.9296875
327 0
328 -5.37109375
329 -0.48828125
330 2.44140625
331 -3.41796875
332 -9.27734375
333 -6.34765625
334 -6.34765625
335 -11.23046875
336 -1.46484375
337 -9.765625
338 -2.9296875
339 -8.7890625
340 -11.23046875
341 -4.39453125
342 -8.30078125
343 -6.34765625
344 -15.13671875
345 -14.16015625
346 -6.8359375
347 -11.23046875
348 -8.7890625
349 2.44140625
350 -1.46484375
351 2.9296875
352 6.34765625
353 1.953125
354 1.953125
355 -9.765625
356 -5.859375
357 -6.34765625
358 -10.25390625
359 -3.41796875
360 -2.9296875
361 -2.44140625
362 3.90625
363 3.90625
364 4.8828125
365 -2.44140625
366 3.90625
367 -5.37109375
368 0.48828125
369 -2.44140625
370 -1.953125
371 -3.41796875
372 0.9765625
373 1.46484375
374 10.25390625
375 9.765625
376 4.8828125
377 7.32421875
378 16.6015625
379 9.765625
380 4.39453125
381 7.32421875
382 -2.44140625
383 2.44140625
384 -5.859375
385 1.953125
386 3.90625
387 -0.9765625
388 -3.41796875
389 2.44140625
390 -3.41796875
391 -4.8828125
392 -6.8359375
393 -8.7890625
394 -4.8828125
395 -1.46484375
396 -2.44140625
397 1.46484375
398 -3.41796875
399 -5.37109375
400 1.953125
401 8.30078125
402 3.41796875
403 5.859375
404 11.71875
405 1.953125
406 4.39453125
407 -1.46484375
408 4.39453125
409 3.41796875
410 -1.46484375
411 3.41796875
412 0
413 -9.27734375
414 -5.859375
415 -9.27734375
416 2.44140625
417 7.8125
418 15.13671875
419 9.765625
420 9.27734375
421 3.90625
422 1.953125
423 10.25390625
424 -0.48828125
425 4.8828125
426 4.8828125
427 3.41796875
428 -0.48828125
429 7.32421875
430 6.34765625
431 6.34765625
432 8.30078125
433 3.90625
434 9.765625
435 8.30078125
436 14.16015625
437 17.08984375
438 14.16015625
439 16.6015625
440 12.20703125
441 13.671875
442 9.765625
443 19.04296875
444 19.53125
445 23.92578125
446 19.53125
447 20.99609375
448 16.6015625
449 15.625
450 12.6953125
451 15.13671875
452 5.859375
453 11.71875
454 8.7890625
455 12.20703125
456 6.8359375
457 4.8828125
458 0.48828125
459 3.90625
460 -0.9765625
461 0
462 3.41796875
463 5.37109375
464 2.44140625
465 5.37109375
466 4.8828125
467 13.18359375
468 9.27734375
469 18.5546875
470 11.23046875
471 16.11328125
472 10.7421875
473 5.859375
474 18.5546875
475 16.11328125
476 25.390625
477 30.76171875
478 30.76171875
479 27.83203125
480 19.04296875
481 21.484375
482 17.08984375
483 17.578125
484 13.18359375
485 14.6484375
486 17.578125
487 18.06640625
488 20.5078125
489 18.5546875
490 16.11328125
491 11.23046875
492 5.859375
493 6.34765625
494 15.13671875
495 6.34765625
496 16.11328125
497 12.6953125
498 15.13671875
499 10.25390625
500 9.765625
501 17.08984375
502 10.7421875
503 10.7421875
504 18.5546875
505 16.11328125
506 14.6484375
507 9.27734375
508 12.6953125
509 13.18359375
510 8.7890625
511 16.6015625
512 15.625
513 11.23046875
514 12.20703125
515 13.18359375
516 13.671875
517 15.13671875
518 10.25390625
519 9.765625
520 12.20703125
521 17.08984375
522 10.25390625
523 6.8359375
524 0.48828125
525 2.9296875
526 8.30078125
527 11.23046875
528 9.27734375
529 11.71875
530 6.8359375
531 12.6953125
532 4.39453125
533 7.32421875
534 15.13671875
535 8.30078125
536 4.8828125
537 5.859375
538 16.6015625
539 20.01953125
540 14.6484375
541 13.18359375
542 11.71875
543 13.671875
544 19.53125
545 15.625
546 16.11328125
547 8.7890625
548 14.16015625
549 17.578125
550 15.625
551 11.71875
552 16.11328125
553 16.11328125
554 13.18359375
555 16.6015625
556 17.578125
557 15.13671875
558 16.6015625
559 15.13671875
560 3.41796875
561 7.8125
562 6.34765625
563 6.8359375
564 9.765625
565 14.6484375
566 22.4609375
567 6.34765625
568 10.25390625
569 11.71875
570 9.27734375
571 13.18359375
572 11.23046875
573 9.765625
574 8.30078125
575 5.859375
576 8.30078125
577 0.48828125
578 10.25390625
579 -3.41796875
580 -18.06640625
581 -7.32421875
582 0.48828125
583 8.7890625
584 20.01953125
585 23.92578125
586 24.4140625
587 13.18359375
588 9.765625
589 8.30078125
590 -2.9296875
591 2.9296875
592 0.48828125
593 3.90625
594 8.7890625
595 8.30078125
596 9.27734375
597 -0.9765625
598 -0.9765625
599 -2.44140625
600 2.9296875
601 1.46484375
602 1.46484375
603 -2.9296875
604 8.7890625
605 2.9296875
606 9.765625
607 10.25390625
608 11.71875
609 11.23046875
610 11.23046875
611 14.16015625
612 5.37109375
613 9.27734375
614 7.8125
615 8.7890625
616 10.7421875
617 13.18359375
618 15.13671875
619 13.671875
620 8.7890625
621 2.44140625
622 4.39453125
623 0.9765625
624 -1.953125
625 0.9765625
626 4.39453125
627 3.41796875
628 10.25390625
629 3.41796875
630 6.8359375
631 3.90625
632 5.859375
633 0.48828125
634 4.8828125
635 10.7421875
636 3.41796875
637 0
638 2.9296875
639 4.39453125
640 -6.34765625
641 4.39453125
642 1.953125
643 1.953125
644 0.48828125
645 0.9765625
646 0.9765625
647 4.8828125
648 -2.9296875
649 5.37109375
650 8.30078125
651 1.46484375
652 -5.37109375
653 0
654 -1.46484375
655 -5.859375
656 -1.46484375
657 0.9765625
658 1.46484375
659 9.765625
660 0.48828125
661 -2.9296875
662 -1.46484375
663 -2.44140625
664 1.953125
665 2.9296875
666 5.859375
667 0.48828125
668 -4.39453125
669 0.9765625
670 4.39453125
671 2.44140625
672 0.9765625
673 2.44140625
674 -12.6953125
675 -4.8828125
676 -12.6953125
677 0.9765625
678 -1.953125
679 5.37109375
680 5.37109375
681 3.41796875
682 4.8828125
683 4.8828125
684 1.953125
685 -3.41796875
686 -0.9765625
687 4.8828125
688 3.41796875
689 -2.44140625
690 2.9296875
691 -9.765625
692 -9.27734375
693 -1.953125
694 -5.37109375
695 1.953125
696 0
697 4.8828125
698 3.41796875
699 -3.90625
700 -1.46484375
701 -9.27734375
702 -12.20703125
703 -11.23046875
704 -6.8359375
705 -7.8125
706 -7.32421875
707 -9.27734375
708 -5.859375
709 -6.8359375
710 -7.32421875
711 -4.39453125
712 -4.8828125
713 -2.9296875
714 2.44140625
715 -4.8828125
716 10.25390625
717 10.25390625
718 2.44140625
719 7.32421875
720 6.8359375
721 10.25390625
722 5.37109375
723 -2.9296875
724 5.37109375
725 -1.953125
726 -0.48828125
727 -8.7890625
728 0.9765625
729 2.9296875
730 -2.44140625
731 0.48828125
732 -1.953125
733 -4.8828125
734 -5.859375
735 -6.34765625
736 -6.8359375
737 -8.30078125
738 -14.16015625
739 -12.20703125
740 -17.578125
741 -13.18359375
742 -8.30078125
743 -4.39453125
744 -3.90625
745 -5.37109375
746 -5.37109375
747 -9.27734375
748 -5.37109375
749 -4.39453125
};
\addplot [green!50.0!black]
table {%
0 2.44140625
1 7.32421875
2 0.9765625
3 5.37109375
4 8.7890625
5 0
6 12.6953125
7 5.37109375
8 10.7421875
9 8.30078125
10 2.9296875
11 6.8359375
12 8.30078125
13 14.16015625
14 16.11328125
15 11.23046875
16 8.30078125
17 9.27734375
18 7.32421875
19 15.13671875
20 10.7421875
21 17.08984375
22 9.765625
23 16.6015625
24 14.6484375
25 3.90625
26 12.20703125
27 7.8125
28 10.25390625
29 7.32421875
30 1.953125
31 4.8828125
32 -4.39453125
33 -6.34765625
34 -2.44140625
35 -8.7890625
36 -6.34765625
37 -1.953125
38 -1.46484375
39 6.34765625
40 8.30078125
41 19.04296875
42 18.5546875
43 9.765625
44 24.4140625
45 23.92578125
46 21.97265625
47 19.53125
48 23.4375
49 17.578125
50 11.23046875
51 13.18359375
52 12.6953125
53 20.5078125
54 17.08984375
55 13.18359375
56 17.578125
57 8.7890625
58 12.20703125
59 12.20703125
60 -2.44140625
61 10.25390625
62 -0.48828125
63 -2.44140625
64 5.859375
65 5.859375
66 8.7890625
67 5.859375
68 1.953125
69 4.8828125
70 -5.37109375
71 -8.30078125
72 -8.30078125
73 0.9765625
74 6.8359375
75 2.9296875
76 11.23046875
77 5.859375
78 7.8125
79 2.44140625
80 6.34765625
81 10.25390625
82 15.13671875
83 16.11328125
84 20.5078125
85 15.13671875
86 8.7890625
87 6.8359375
88 6.8359375
89 6.34765625
90 2.9296875
91 8.30078125
92 5.859375
93 4.39453125
94 12.6953125
95 2.9296875
96 3.41796875
97 2.9296875
98 -7.8125
99 -0.48828125
100 0
101 7.32421875
102 7.32421875
103 3.90625
104 9.27734375
105 9.765625
106 2.9296875
107 3.90625
108 5.859375
109 11.23046875
110 0
111 2.44140625
112 -19.04296875
113 -7.8125
114 11.23046875
115 4.39453125
116 13.671875
117 19.53125
118 12.6953125
119 9.27734375
120 3.41796875
121 9.27734375
122 10.25390625
123 14.6484375
124 11.23046875
125 7.32421875
126 18.06640625
127 8.30078125
128 11.71875
129 17.08984375
130 9.27734375
131 10.7421875
132 7.32421875
133 10.7421875
134 9.27734375
135 1.46484375
136 2.44140625
137 -9.27734375
138 0
139 7.32421875
140 -1.46484375
141 8.7890625
142 5.859375
143 -1.953125
144 9.27734375
145 -1.953125
146 6.8359375
147 -5.37109375
148 8.7890625
149 7.32421875
150 4.8828125
151 19.04296875
152 9.27734375
153 10.25390625
154 11.71875
155 7.32421875
156 10.7421875
157 9.27734375
158 6.8359375
159 0
160 -10.25390625
161 9.27734375
162 19.53125
163 18.06640625
164 23.4375
165 27.83203125
166 22.94921875
167 16.11328125
168 23.4375
169 19.53125
170 21.484375
171 23.92578125
172 7.32421875
173 16.6015625
174 19.53125
175 6.34765625
176 12.6953125
177 13.18359375
178 4.39453125
179 0
180 -6.8359375
181 -8.7890625
182 -10.7421875
183 -7.32421875
184 -8.30078125
185 -5.37109375
186 0.9765625
187 -12.20703125
188 3.90625
189 -0.9765625
190 0
191 2.9296875
192 -4.8828125
193 1.46484375
194 1.953125
195 -0.48828125
196 -3.90625
197 -0.9765625
198 -0.9765625
199 -4.39453125
200 -5.859375
201 -1.953125
202 -8.7890625
203 -3.90625
204 1.46484375
205 -3.90625
206 -0.48828125
207 -3.90625
208 -0.9765625
209 3.41796875
210 -3.41796875
211 1.953125
212 5.37109375
213 5.37109375
214 10.25390625
215 4.8828125
216 6.34765625
217 7.8125
218 5.859375
219 6.34765625
220 5.859375
221 7.8125
222 1.46484375
223 10.25390625
224 5.37109375
225 1.953125
226 8.30078125
227 2.44140625
228 4.39453125
229 9.765625
230 6.8359375
231 16.11328125
232 12.6953125
233 3.41796875
234 3.90625
235 7.32421875
236 9.765625
237 6.34765625
238 6.34765625
239 0.9765625
240 -2.44140625
241 0.9765625
242 -2.9296875
243 4.8828125
244 0.9765625
245 1.46484375
246 -0.48828125
247 -3.41796875
248 -0.9765625
249 4.8828125
250 -14.16015625
251 -0.9765625
252 -8.7890625
253 -14.6484375
254 -2.9296875
255 -20.01953125
256 -1.953125
257 -14.6484375
258 -10.7421875
259 -4.39453125
260 2.44140625
261 3.90625
262 -4.8828125
263 -3.41796875
264 -2.9296875
265 -7.8125
266 -0.48828125
267 -1.46484375
268 -3.41796875
269 -4.39453125
270 0
271 4.39453125
272 2.9296875
273 -1.46484375
274 11.23046875
275 1.46484375
276 6.34765625
277 1.46484375
278 0.48828125
279 6.8359375
280 -0.48828125
281 3.90625
282 2.44140625
283 1.46484375
284 4.39453125
285 -6.34765625
286 1.953125
287 -1.46484375
288 -3.90625
289 -1.46484375
290 1.46484375
291 5.37109375
292 5.37109375
293 -2.9296875
294 2.9296875
295 0.9765625
296 2.9296875
297 -3.90625
298 -1.46484375
299 -0.48828125
300 -2.9296875
301 -2.9296875
302 0.9765625
303 3.90625
304 0.9765625
305 -1.46484375
306 -6.8359375
307 -1.953125
308 1.46484375
309 0.9765625
310 1.46484375
311 2.44140625
312 3.41796875
313 -4.39453125
314 2.44140625
315 0.48828125
316 -2.9296875
317 -4.39453125
318 -3.41796875
319 -0.48828125
320 -4.39453125
321 -0.48828125
322 0.9765625
323 3.41796875
324 10.25390625
325 10.25390625
326 16.6015625
327 9.765625
328 2.9296875
329 -0.48828125
330 -1.953125
331 -2.9296875
332 -2.44140625
333 5.859375
334 4.39453125
335 9.27734375
336 17.08984375
337 0
338 5.37109375
339 10.7421875
340 -0.9765625
341 8.30078125
342 6.34765625
343 5.37109375
344 -0.9765625
345 -3.41796875
346 3.41796875
347 -3.90625
348 -4.8828125
349 4.39453125
350 1.46484375
351 14.6484375
352 9.27734375
353 7.32421875
354 5.859375
355 -5.859375
356 -3.90625
357 -7.32421875
358 0
359 3.41796875
360 -0.9765625
361 -0.9765625
362 -0.48828125
363 -1.46484375
364 3.41796875
365 0
366 5.859375
367 -5.37109375
368 -3.90625
369 -7.32421875
370 -10.25390625
371 -1.46484375
372 0
373 2.44140625
374 7.32421875
375 2.9296875
376 -2.44140625
377 -7.8125
378 11.71875
379 11.71875
380 4.8828125
381 8.7890625
382 -7.32421875
383 -0.48828125
384 -6.34765625
385 -0.9765625
386 5.859375
387 0.48828125
388 0.48828125
389 1.953125
390 -2.44140625
391 -6.34765625
392 -7.8125
393 -9.765625
394 -0.9765625
395 -8.7890625
396 4.8828125
397 4.8828125
398 0.48828125
399 4.8828125
400 2.44140625
401 5.859375
402 1.953125
403 3.90625
404 6.34765625
405 -2.9296875
406 7.32421875
407 -2.44140625
408 -1.953125
409 -0.48828125
410 -5.859375
411 4.8828125
412 -2.44140625
413 -5.859375
414 -6.34765625
415 -7.32421875
416 -0.48828125
417 -8.30078125
418 10.25390625
419 6.34765625
420 3.41796875
421 3.41796875
422 -3.90625
423 7.32421875
424 4.8828125
425 7.32421875
426 12.20703125
427 11.23046875
428 4.8828125
429 6.8359375
430 -3.41796875
431 3.90625
432 1.46484375
433 -2.9296875
434 7.32421875
435 1.953125
436 6.8359375
437 11.23046875
438 12.20703125
439 13.671875
440 9.27734375
441 12.20703125
442 -5.859375
443 10.7421875
444 12.20703125
445 17.578125
446 19.04296875
447 15.625
448 8.7890625
449 5.859375
450 6.8359375
451 7.32421875
452 -1.953125
453 9.27734375
454 1.46484375
455 2.44140625
456 0.9765625
457 -1.953125
458 -2.44140625
459 -2.9296875
460 -6.34765625
461 3.41796875
462 -3.41796875
463 -1.953125
464 -2.44140625
465 -0.48828125
466 -3.90625
467 4.8828125
468 9.765625
469 14.16015625
470 2.9296875
471 4.8828125
472 -6.34765625
473 -1.46484375
474 7.8125
475 5.859375
476 19.53125
477 16.6015625
478 13.18359375
479 1.46484375
480 -3.41796875
481 3.90625
482 -0.9765625
483 0.48828125
484 -1.953125
485 0.48828125
486 3.90625
487 -0.48828125
488 7.8125
489 4.39453125
490 -5.37109375
491 -10.25390625
492 -9.765625
493 -13.671875
494 -3.41796875
495 -7.8125
496 -4.8828125
497 -3.41796875
498 -0.48828125
499 -4.39453125
500 -2.44140625
501 8.30078125
502 -0.9765625
503 0
504 5.859375
505 -5.859375
506 1.953125
507 -17.578125
508 -1.46484375
509 6.34765625
510 -5.37109375
511 10.7421875
512 3.41796875
513 0
514 3.90625
515 -4.8828125
516 6.34765625
517 6.8359375
518 1.953125
519 2.9296875
520 1.46484375
521 10.7421875
522 1.46484375
523 0
524 -11.23046875
525 -7.8125
526 0.9765625
527 -6.34765625
528 1.953125
529 6.8359375
530 -2.44140625
531 1.953125
532 -7.32421875
533 -3.41796875
534 0
535 -7.8125
536 -8.7890625
537 -8.7890625
538 -0.48828125
539 1.953125
540 -5.859375
541 -1.953125
542 -7.8125
543 -4.39453125
544 3.41796875
545 -2.44140625
546 -0.48828125
547 0.48828125
548 1.953125
549 11.23046875
550 6.8359375
551 6.8359375
552 9.27734375
553 3.41796875
554 12.20703125
555 10.25390625
556 18.06640625
557 16.6015625
558 16.11328125
559 16.11328125
560 6.8359375
561 13.18359375
562 8.30078125
563 5.859375
564 12.6953125
565 16.6015625
566 24.4140625
567 9.27734375
568 10.25390625
569 9.765625
570 7.32421875
571 10.7421875
572 10.25390625
573 16.11328125
574 14.6484375
575 5.37109375
576 9.765625
577 4.8828125
578 13.18359375
579 -1.46484375
580 -7.32421875
581 8.7890625
582 9.27734375
583 13.671875
584 27.83203125
585 20.99609375
586 6.34765625
587 2.9296875
588 0
589 5.859375
590 -1.953125
591 8.30078125
592 3.90625
593 6.8359375
594 9.765625
595 2.9296875
596 -0.48828125
597 2.44140625
598 -3.41796875
599 0
600 4.8828125
601 0.9765625
602 -9.27734375
603 -10.25390625
604 1.46484375
605 -7.32421875
606 4.39453125
607 0
608 2.44140625
609 3.90625
610 6.8359375
611 8.7890625
612 0
613 -2.44140625
614 0.48828125
615 -4.39453125
616 -5.859375
617 0
618 2.44140625
619 5.37109375
620 -0.48828125
621 -0.48828125
622 0.48828125
623 -3.90625
624 -0.9765625
625 -3.90625
626 6.34765625
627 0.48828125
628 4.39453125
629 4.8828125
630 7.32421875
631 5.37109375
632 -0.48828125
633 -2.9296875
634 1.953125
635 9.765625
636 10.25390625
637 5.859375
638 11.23046875
639 10.7421875
640 1.46484375
641 7.32421875
642 5.37109375
643 2.44140625
644 12.20703125
645 5.37109375
646 8.30078125
647 11.71875
648 3.41796875
649 12.6953125
650 7.8125
651 9.27734375
652 1.953125
653 5.37109375
654 6.8359375
655 0.9765625
656 4.39453125
657 8.30078125
658 5.859375
659 9.765625
660 1.953125
661 15.13671875
662 10.25390625
663 12.20703125
664 16.6015625
665 15.13671875
666 17.578125
667 2.9296875
668 1.46484375
669 12.6953125
670 9.765625
671 11.23046875
672 10.7421875
673 20.01953125
674 4.8828125
675 -2.9296875
676 0.48828125
677 4.39453125
678 0
679 5.37109375
680 9.27734375
681 9.765625
682 -0.9765625
683 2.9296875
684 6.34765625
685 -3.90625
686 2.44140625
687 9.765625
688 7.32421875
689 3.90625
690 3.41796875
691 -2.44140625
692 -4.39453125
693 0.48828125
694 7.32421875
695 8.7890625
696 9.765625
697 5.37109375
698 9.27734375
699 6.8359375
700 2.9296875
701 -1.46484375
702 0.48828125
703 1.46484375
704 1.953125
705 -0.48828125
706 -0.9765625
707 -12.6953125
708 -4.8828125
709 -1.46484375
710 -8.30078125
711 0.48828125
712 -0.48828125
713 -5.859375
714 -4.8828125
715 -7.8125
716 9.27734375
717 5.37109375
718 2.44140625
719 9.765625
720 1.953125
721 4.39453125
722 2.44140625
723 -5.859375
724 8.7890625
725 3.41796875
726 6.8359375
727 -3.90625
728 4.8828125
729 8.30078125
730 3.90625
731 13.671875
732 9.765625
733 9.27734375
734 6.34765625
735 2.44140625
736 1.46484375
737 -3.41796875
738 4.39453125
739 5.37109375
740 0
741 -0.48828125
742 3.90625
743 2.9296875
744 3.90625
745 2.9296875
746 6.8359375
747 2.44140625
748 7.8125
749 2.9296875
};
\end{groupplot}

\end{tikzpicture}
	\end{adjustbox}
	\vspace{-2em}
	\label{fig:oacl-signals}
	\caption{Smoothed and artifact signal superimposed on the raw EEG signal for a single channel in part of a single trial.}
\end{figure*}
\section{Ocular Artifact Correction}
The OACL method consists of two parts. First, the raw EEG signals are processed to obtain the artifact signals, representing the parts of the signal that are artifacts. Then we find the \emph{filtering parameter} $\Theta$ for each the artifact signals, which determines the amount of each signal that are to be removed from the raw signal in order to obtain the corrected signal. 
Since the original OACL method \citep{li2015ocular} was devised for binary class datasets, we generalize the algorithm for multi-class datasets. 

Let $x = \{s_0, ...,s_n\}$ where $s_t \in \mathbb{R}_{\geq 0}$ denote the raw EEG data for some arbitrary channel. For simplicity, we can interpret $x$ as a function $x : \mathbb{N}_{\geq 0} \rightarrow \mathbb{R}_{\geq 0}$ where $x(t)$ denotes the amplitude of $x$ at time $t$. 
From $x(t)$ we perform all steps in artifact detection and removal.

%Let $t \in \mathbb{N}_{\geq 0}$ be the time of a EEG sample, then $x(t)$ denotes the amplitude measured at time $t$, that is, $x(t)$ denotes the raw EEG signal for some channel.

\subsection{Artifact Detection}
The goal of artifact detection is to find the artifact signal $a(t)$ from $x(t)$ that represents which parts of $x(t)$ that contains ocular artifactsare noise. Before finding the artifact signal $as(t)$, we first obtain a smoothed signal by applying a \emph{moving average filter} to $x(t)$, in order to smooth out short-term fluctuations from e.g. external interference. A moving average filter is a function $s: \mathbb{N} \rightarrow \mathbb{R}$ that given time t, computes the average of the $\frac{m}{2}$ samples on either side of the tth sample. 
\begin{equation}
\label{eq:movavg}
s(t) = \frac{1}{m}\sum_{t-\frac{m}{2}}^{t+\frac{m}{2}}x(t)
\end{equation}
where $m$ is the number of neighboring points used. The smoothed signal $s(t)$ shown together with the raw signal $x(t)$ are illustrated in \cref{fig:oacl-signals}.

From the smoothed signal $s(t)$ the OACL methodOACL finds the changes in amplitude between the samples, which could indicate eye movement. Therefore, we proceed by computing the relative heights between samples as the maximal difference in amplitude between a sample at time $t$ and its neighboring samples.
\begin{align*}
\Delta (t) = max(&|s(t)-s(t-1)|,\\
&|s(t+1) - s(t)|) \numberthis \label{eq:relheights}
\end{align*}
where we can consider $\Delta(t)$ as the describing the fluctuations in the signal.

Now, we want to have some measure of what an artifact signal looks like. \citet{li2015ocular} found by inspection that ocular artifacts generally occur with sudden changes in amplitude ($\Delta$) between $[30\mu V-50\mu V]$ and $[70\mu V-150\mu V]$. The problem with this approach is, that it does not generalize to data collected using different setups for the EEG measurement. For this reason, we automatically estimate the ranges by considering them parameters to be optimized through Bayesian Optimization, discussed in \cref{sec:bayesian-optimization}.
For now, assume that we have some arbitrary ranges
\begin{equation}\label{ranges}
R=[l, u] \quad  l,u \in \mathbb{N}_{\geq 0}
\end{equation}
then we can find the sample indexes $t$ where $\Delta (t)$  lies in the range $R$:
\begin{align*}
P = \{t \quad | \quad &\frac{m}{2} < t < n-\frac{m}{2}  \quad \textnormal{and} \\
& l < \Delta (t) < u\} \numberthis \label{eq:peaks}
\end{align*}
where $P$ is the indexes of the samples of the smoothed signal, where a change in amplitude lies within the range that characterizes an ocular artifact.

We now have the peaks of the ocular artifacts in $s(t)$, but we still need all of the artifact before we can correct it. The approach in the OACL method is to define the artifact to be from the closest zero point to the peak, before the peak, to the closest zero point after the peak. We can now find the artifact signal 
\begin{equation}
\label{eq:artifactsignal}
a(t) =
\begin{cases}
s(t)      & \quad \text{if } z_b \leq t < z_a\\
0  & \quad \text{otherwise}\\
\end{cases}
\end{equation}
where $z_b, z_a$ are the zero points after and before respectively. The concept of zero points of an artifact are illustrated in \cref{fig:zero-points}figure \todo{modify figure to show the concept of zero points}.
\todo{explain how we find zero points emil}
Recalling that we can use an arbitrary number of ranges in \cref{eq:ranges}, we obtain the set of artifact signals for a single channel of a single trial as:
\begin{equation}\label{eq:artifact-signals}
A(t)=  \begin{pmatrix}
a_1(t) \\
a_2(t) \\
\vdots  \\
a_n(t) 
\end{pmatrix}
\end{equation}
The idea behind this, is that different types of artifacts can be found by using different kinds of ranges. 
\subsection{Artifact Removal}
With the artifact signal $a(t)$ extracted from the raw EEG data the next task is to remove the artifacts characterized by $a(t)$ from the original signal $x(t)$. Since signals are closed under subtraction we can obtain the corrected signal by subtracting the artifact signal from the raw EEG signal:
\begin{equation}\label{eq:corrected-signal}
c(t) = x(t) - \theta A(t)
\end{equation}
where $\theta \in \mathbb{R}_{\geq 0}$ is the \emph{filtering parameter}. Intuitively, $\theta$ is a factor that determines the "amount" of the artifact signal to subtract from the raw signal. 

The original approach of OACL is to obtain the $\theta$ parameter by training a binary logistic regression classifier with a modified hypothesis function with the latent variable derived from the power of the raw signal. Instead of this approach, we consider $\theta$ parameters as hyperparameters to the ocular artifact detection, which means that we can find the values for $\theta$ by optimizing them through the Bayesian Optimization algorithm. In short, this means that find the corrected signals for one subject as
\begin{align}\label{eq:corrected-signal}
C(t)=  \begin{pmatrix}
c_1(t) \\
\vdots  \\
c_{k}(t) 
\end{pmatrix}
\end{align}
where $kY$ is the number of channels of classes. By \cref{eq:corrected-signal} this means that we optimize over $k \times n$ $\theta$ parameters in bayesian optimization, where $n$ is the number of artifact sigals in \cref{eq:artifact-signals}. As we expect to remove 0-100\% of the artifact signal from the raw signal, we constraint the domain of the $\theta$ parameter to be between 0 and 1.  

