\section{Experimental Results}
In order to evaluate our method, we use the BCI Competition IV dataset 2a \citep{brunner2008bci}, which contains 4-class motor imagery EEG data from 9 subjects. The dataset consist of labeled training and test sets. Each subject participated in two sessions of 6 runs on different days. A run consist of 48 labeled trials, divided evenly between the 4 classes. Each trial measured the brain signals of a subject on 22 EEG channels and 3 EOG channels. We discarded the EOG channels since we are interested in correcting artifacts without any reference signals.

We set up two pipelines to be compared. The first consists of OACL, followed by FBCSP and Random Forest. The second is without the OACL step, i.e., just FBCSP and Random Forest. We perform 6-fold cross-validation on the training data using five runs for training and one for validation. We run 200 iterations of Bayesian Optimization of select hyperparameters for each pipeline. The best hyperparameters for each pipeline is then used for the final evaluation. We repeat this for each subject. \Cref{fig:results} shows the obtained accuracies and KAPPA scores for each subject.

\begin{table}[H]
	\centering
	\caption{Accuracy and KAPPA score for each subject.}
	\label{fig:results}
	\begin{tabular}{@{}ccc|cc@{}}
		\toprule
		\textbf{S}             & \multicolumn{2}{c|}{\textbf{W. BO OACML}} & \multicolumn{2}{c}{\textbf{W/O BO OACML}} \\ \midrule
		\multicolumn{1}{c|}{}  & Acc                   & KAPPA             & Acc                   & KAPPA             \\ \midrule
		\multicolumn{1}{c|}{1} &                       &                   &                       &                   \\
		\multicolumn{1}{c|}{2} &                       &                   &                       &                   \\
		\multicolumn{1}{c|}{3} &                       &                   &                       &                   \\
		\multicolumn{1}{c|}{4} &                       &                   &                       &                   \\
		\multicolumn{1}{c|}{5} & \textbf{}             &                   & \textbf{}             &                   \\
		\multicolumn{1}{c|}{6} &                       &                   &                       &                   \\
		\multicolumn{1}{c|}{7} &                       &                   &                       &                   \\
		\multicolumn{1}{c|}{8} &                       &                   &                       &                   \\
		\multicolumn{1}{c|}{9} &                       &                   &                       &                   \\ \bottomrule
	\end{tabular}
\end{table}

To determine the significance of these results we use the Wilcoxon signed-rank test. \Cref{fig:wilcoxon} shows the results of the test.

\begin{table}[H]
	\centering
	\caption{Wilcoxon signed-rank test}
	\label{fig:wilcoxon}
	\begin{tabular}{@{}l|llll@{}}
		\toprule
		S & No OACL & OACL & Diff & Rank \\ \midrule
		&               &                 &      &      \\
		&               &                 &      &      \\
		&               &                 &      &      \\
		&               &                 &      &      \\
		&               &                 &      &      \\
		&               &                 &      &      \\
		&               &                 &      &      \\
		&               &                 &      &      \\
		&               &                 &      &      \\ \bottomrule
	\end{tabular}
\end{table}

\subsection{Discussion}
Here we discuss the results given in section 3, and talk more about what the results imply/how it could be improved.
Remember to discuss the issues of complexity of experiment vs. how much effort we put into selecting the "best" candidate in Bayesian Optimization