\section{Introduction}
The field of Brain-Computer Interfaces (BCI) has in the recent years been under active research, especially with the popularity of machine learning techniques. The reason for the interest, is the many useful application of a well-working BCI, such as replacing lost motor function in disabled people, helping with analysis in brain imaging to diagnose brain conditions or novel applications in computer games. 

The general idea of a BCI is to measure brain activity represented by electroencephalogram (EEG) signals, by putting sensors on the scalp, which can measure the electric impulses. However, the EEG data is noisy at best, and this problem can severely affect the classification results in a machine learning algorithm. Therefore, signal processing is an important step in any given BCI.

All in all, this leaves us with several steps in which several techniques may be applied to obtain the corrected EEG signal. Each technique applied may require any number of parameters to be tuned for obtaining the optimal results. Users of the BCI or medical professionals are usually not knowledgeable about tuning such parameters, hence requires either an export to help determine them or extensive training.
Another, more useful approach would be to automatically infer the hyper-parameters from the training data. Recent work about algorithmically optimizing machine learning parameters has seen popularity by using Bayesian Optimization, in which the learning process is seen as a black-box function which parameters can be optimized.

\subsection{Related Work}
Much research effort has been put into developing or applying techniques for noise/artifact correction in EEG signals. Well-known methods, such as PCA, ICA or DWT, in the signal processing world has had mixed results.
% EEG full of noise --> affects classificationr esults
% General noise reduction methods has been used for all types of artifacts. PCA, ICA, DWT etc...
% Someone proposed to make several passes over eeg to remove a single type of artifact
% oacl guys make oacl, also inspired from ecg/whatever method.
