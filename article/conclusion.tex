\section{Conclusion}
We considered the negative effect of ocular artifacts on classification of EEG 4-class motor imagery. As a possible solution to the problem, we presented a method for ocular artifact correction in multi class EEG data, as an extension to the binary class ocular artifact correction method OACL\citep{li2015ocular}. Furthermore, we showed that the hyperparameters that was manually determined in OACL, can be automatically tuned by application of Bayesian Optimization algorithm. By testing the ocular artifact correction on the BCI Competition dataset IV2a, we  we found that the best hyperparameters found after 200-300 iterations did not improve classification performance compared to classification without application of the ocular artifact correction method. We discuss that reasons for this include the problematically large optimization space over filtering parameters and that filtering parameters may not be generalizable over several trials.

Possible improvements for future work are adapting the logistic regression approach used by \citep{li2015ocular} for multi-class estimation of filtering parameters. Secondly, the detection and correction of the residual as a subprocess to the main artifact correction process. 
%Recapitulates the problem and the contribution.
%Assesses the significance of the contribution.
%Suggests and outlines future work, open problems, etc.

\section{Acknowledgement}
We thank Felipe Soares da Costa for valuable feedback. Furthermore we would like to thank Aalborg University for lending us a server, with the purpose of testing program configurations.  