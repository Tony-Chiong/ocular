\begin{abstract}

\noindent Brain-Computer Interfaces (BCI) is becoming increasingly useful for things such as diagnosing brain conditions and restoring motor function in disabled people. One area that still needs to improve is artifact correction. In this study, a state-of-the-art artifact correction method dubbed OACL is expanded upon in order to make it more generally applicable. The new method (MCOACL) is a multi-class version of OACL that does not require expert input. We use Filter Bank Common Spatial Pattern (FBCSP) for feature extraction, followed by Random Forest (RF) for classification. To evaluate the method it is put in a pipeline A consisting of MCOACL $\to$ FBCSP $\to$ RF, which is then compared to an identical pipeline B without MCOACL. The hyperparameters for these pipelines are optimized using Bayesian Optimization (BO). The 4-class EEG data from dataset 2a of BCI Competition IV, comprising 22 EEG channels and 2 sessions of 9 subjects over 6 runs, is used to evaluate the method. The results from pipeline A could not show statistically significance compared to pipeline B. The result might however stem from parameters not being optimized correctly, which is discussed in the section \cref{sec:discussion}.

\end{abstract}