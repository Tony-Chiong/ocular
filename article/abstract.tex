\begin{abstract}

\noindent Brain-Computer Interfaces (BCI) is becoming increasingly useful for things such as diagnosing brain conditions and restoring motor function in disabled people. One area that still needs to improve is artifact correction. In this study, a state-of-the-art artifact correction method OACL\citep{li2015ocular} is extended in order to make it generally applicable. The new method is a multi-class version of OACL that does not require expert input. We use Filter Bank Common Spatial Pattern for feature extraction, followed by Random Forest for classification. To evaluate the method it is put in a pipeline A, consisting of Ocular Artifact Correction, Feature Extraction and Classification. The classification results from this pipeline is then compared a similar pipeline without ocular artifact correction. The hyperparameters for the methods used in the pipelines are optimized using Bayesian Optimization. The 4-class EEG data from dataset 2a of BCI Competition IV, comprising 22 EEG channels and 2 sessions of 9 subjects over 6 runs, is used to evaluate the pipelines. The results from pipeline with ocular artifact correction did not yield significantly better classification results than the pipeline with no artifact correction. The result might however stem from parameters not being optimized correctly.

\end{abstract}