\begin{abstract}

\noindent Brain-Computer Interfaces (BCI) is becoming increasingly useful for things such as diagnosing brain conditions and restoring motor function in disabled people. One area that still needs to improve is artifact correction. In this study, a state-of-the-art artifact correction method dubbed OACL is expanded upon in order to make it more generally applicable. The new method is a multi-class version of OACL that does not require expert input. We use Filter Bank Common Spatial Pattern (FBCSP) for feature extraction, followed by Random Forest (RF) for classification. To evaluate the method it is put in a pipeline A consisting of OACL $\to$ FBCSP $\to$ RF, which is then compared to an identical pipeline B without OACL. The hyperparameters for these pipelines are optimized using Bayesian Optimization (BO). The 4-class EEG data from dataset 2a of BCI Competition IV, comprising 22 EEG channels and 2 sessions of 9 subjects over 6 runs, is used to evaluate the method. Pipeline A yielded a kappa coefficient of 0.??? and classification accuracy of ??.?\%, while pipeline B was slightly ?lower/higher? at 0.??? and ??.?\%. The results are ?not? statistically significant since a two-tailed Wilcoxon signed-rank test resulted in a p-value of ?.???, which is ?less/greater? than the significance level set to 0.05.

\todo{Fill out information in abstract.}
\todo{Consider: cut down on the details.}

\end{abstract}